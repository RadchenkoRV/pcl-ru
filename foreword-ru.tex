\chapter*{От коллектива переводчиков}
\addcontentsline{toc}{chapter}{От коллектива переводчиков}

\thispagestyle{empty}

В своих руках вы держите перевод очень интересной книги, впервые опубликованной почти
десять лет назад.  Её перевод начался тоже очень давно~-- почти семь лет назад, сначала в
форме wiki, с помощью которого координировался процесс перевода и выкладывался переведенный
текст.  Несколько лет позже, когда была переведена большая часть этой книги, процесс
замедлился, но от издательства <<ДМК Пресс>> поступило предложение издать перевод в
бумажном виде с сохранением перевода в открытом виде.  Результат вы держите в своих
руках\footnote{Если вы заметите ошибку или неточность в переводе, то пожалуйста, сообщите
  о ней на сайте проекта (\url{https://github.com/pcl-ru/pcl-ru}).}\hspace{\footnotenegspace}.

За время, прошедшее с момента начальной публикации оригинала, произошло множество событий~--
функциональное программирование становится мейнстримом, появляются новые языки, такие как
Clojure\pclfootnote{Clojure~-- Lisp-подобный язык для JVM, ставший достаточно популярным}, да
и мир Common Lisp не стоит на месте: появляются библиотеки, улучшаются свободные
реализации, такие как SBCL, и т.д.  Lisp и его диалекты достаточно активно используются в
коммерческом программировании: так, например, Google купил компанию ITA Software, которая
использовала Common Lisp для разработки своих продуктов.

Информацию о Common Lisp можно найти и на русском языке.
\mbox{\url{http://lisper.ru/}}~-- один из основных русскоязычных сайтов, посвященных Lisp.
В~\pclURL{http://fprog.ru/planet/}{<<Русскую планету функционального программирования>>}
входит несколько блогов, авторы которых пишут о Common Lisp.  Вы также можете интерактивно
пообщаться с любителями Lisp в Jabber-конференции \verb|lisp@conference.jabber.ru|.

Я хочу поблагодарить всех, кто принимал участие в подготовке данного издания~-- переводил,
вычитывал и всячески помогал улучшать этот текст: Павла Алхимова, Стаса Бокарёва, Ивана
Болдырева, Виталия Брагилевского, Александра Данилова, Дениса Дереку, Александра
Дергачева, Сергея Дзюбина, Кирилла Горкунова, Алексея Замкового, Петра Зайкина, Илью
Звягина, Евгения Зуева, Андрея Карташова, Виталия Каторгина, Сергея Катревича, Евгения
Кирпичёва, Алекса Классика, Максима Колганова, Кирилла Коринского, Андрея Москвитина,
Константина Моторного, Дмитрия Неверова, Дмитрия Панова, Сергея Раскина, Арсения
Слободюка, Арсения Солокха, Михайло Сорочана, Илью Струкова, Александра Трусова, Валерия
Федотова, Михаила Шеблаева, Михаила Шевчука, Вадима Шендера, Ивана Яни.

От коллектива переводчиков я бы хотел поблагодарить издательство <<ДМК Пресс>> за
организацию официального оформления перевода и предоставление ресурсов, которые позволили
сильно улучшить перевод.

\vspace{1em}

Приятного чтения!

\begin{flushright}
  От коллектива переводчиков, Алекс Отт
\end{flushright}

%%% Local Variables:
%%% mode: latex
%%% TeX-master: "pcl-ru"
%%% End:
