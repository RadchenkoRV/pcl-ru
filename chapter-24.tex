\chapter{Практика. Разбор двоичных файлов}
\label{ch:24}

\thispagestyle{empty}

В этой главе я покажу вам, как создать библиотеку, которую мы сможем использовать при
написании кода для чтения и записи двоичных файлов. Мы воспользуемся этой библиотекой в
главе~\ref{ch:25} для написания программы разбора тегов ID3, механизма, используемого для
хранения метаданных в файлах MP3, таких как исполнитель и название альбома. Эта библиотека
является также примером использования макросов для расширения языка новыми конструкциями,
превращая его в язык специального назначения, предназначенный для решения специфических
задач, в данном случае чтения и записи двоичных файлов. Так как мы разработаем эту
библиотеку за один присест, включая несколько промежуточных версий, вам может показаться,
что мы напишем очень много кода. Но после того, как все будет сказано и сделано, вся
библиотека целиком займёт менее 150 строк кода, а самый большой макрос будет длиной всего
в 20 строк.

\section{Двоичные файлы}

На достаточно низком уровне абстракции все файлы являются <<двоичными>> в том смысле, что
они просто содержат набор чисел, закодированных в двоичной форме.  Однако обычно
различают \textit{текстовые файлы}, в которых все числа могут быть интерпретированы как
знаки, представляющие человекочитаемый текст, и \textit{двоичные файлы}, которые содержат
данные, которые при интерпретации их как знаков выдают непечатаемые знаки\footnote{В
  ASCII первые 32 знака являются непечатаемыми \textit{управляющими знаками}, изначально
  использовавшимися для управления работой телетайпа, и осуществляющими такие действия, как
  выдача звукового сигнала, возврат на один знак назад, перемещение на следующую строку,
  возврат каретки в начало строки. Из этих 32 управляющих знаков только три, знак новой
  строки, возврат каретки и горизонтальная табуляция, типичны для текстовых файлов.}.

Двоичные форматы файлов обычно проектируются в целях повышения компактности данных и
эффективности их разбора~--- это и является их главным преимуществом над текстовыми
форматами. Для достижения этих критериев двоичные файлы обычно имеют такую структуру на
диске (on-disk structures), которая легко отображается на структуры данных, используемые
программой для представления в памяти хранящихся в файлах данных\footnote{Некоторые
  форматы двоичных файлов сами являются структурами данных в памяти (in-memory data
  structures): во многих операционных системах существует возможность отображения файла в
  память, и низкоуровневые языки, такие как C, могут рассматривать область памяти,
  содержащую данные файла, так же, как и любую другую область памяти; данные, записанные в
  эту область памяти, сохраняются в нижележащий файл при отключении его отображения в
  память. Однако такие форматы файлов являются платформеннозависимыми, так как
  представление в памяти даже таких простых типов данных, как числа, зависит от
  аппаратного обеспечения, на котором выполняется программа. Поэтому любой формат файла,
  претендующий на платформеннонезависимость, должен определять каноническое представление
  всех используемых им типов данных, которое может быть отображено в представление в
  памяти фактических данных для определённого вида машин или для определённого языка.}.

Разработанная библиотека предоставит нам простой способ описания соответствия между
структурами на диске, определённых двоичным форматом файла, и объектами Lisp в
памяти. Использование этой библиотеки cделает лёгким написание программ, осуществляющих
чтение двоичных файлов, преобразование их в объекты Lisp для дальнейших манипуляций и
запись в другой двоичный файл.

\section{Основы двоичного формата}

Начальной точкой в чтении и записи двоичных файлов является открытие файла для чтения и
записи отдельных байтов. Как я описывал в главе~\ref{ch:14}, и \lstinline{OPEN}, и
\lstinline{WITH-OPEN-FILE} принимают ключевой аргумент \lstinline{:element-type}, который
устанавливает базовую единицу передачи данных для потока. Для работы с двоичными файлами
нужно указать \lstinline{(unsigned-byte 8)}. Входной поток, открытый с таким параметром
\lstinline{:element-type}, будет возвращать числа от 0 до 255 при каждой его передаче в вызов
\lstinline{READ-BYTE}. И наоборот, мы можем записывать байты в выходной поток с типом
элементов \lstinline{(unsigned-byte 8)} путём передачи чисел от 0 до 255 в \lstinline{WRITE-BYTE}.

Выше уровня отдельных байтов большинство двоичных форматов используют минимальное
количество примитивных типов данных: различные представления чисел, текстовые строки,
битовые поля и т.~д., которые затем комбинируются в более сложные структуры. Поэтому
вашим первым заданием будет определение каркаса для написания кода чтения и записи
примитивных типов данных, используемых данным двоичным форматом.

В качестве простого примера представим, что мы имеем дело с двоичным форматом, который
использует беззнаковые 16-битные целые числа в качестве примитивного типа данных. Для
осуществления чтения таких целых нам нужно прочитать два байта, а затем скомбинировать их
в одно число путём умножения одного байта на 256 (то есть \lstinline!2^8!) и добавления к
нему второго байта. Предположив, например, что двоичный формат определяет хранение таких
16-битных сущностей в \textit{обратном порядке байтов}
(\textit{big-endian})\footnote{Термин \textit{big-endian} и его противоположность
  \textit{little-endian} заимствованы из \textit{Путешествий Гулливера} Джонатана Свифта и
  указывают на способ представления многобайтового числа упорядоченной последовательностью
  байт в памяти или в файле. Например, число 43~981, abcd в шестнадцатеричной системе
  счисления, представленное как 16-битная сущность, состоит из двух байт: ab и cd. Для
  компьютера не имеет значения, в каком порядке эти два байта хранятся, пока все следуют
  этому соглашению. Конечно, в случае, когда определённый выбор должен быть сделан между
  двумя одинаково хорошими вариантами, одна вещь, в которой вы можете не сомневаться,~--- это
  то, что не все будут согласны. Для получения большей, чем вы даже хотите знать,
  информации и для того, чтобы увидеть? где термины \textit{bin-endian} и
  \textit{little-endian} были впервые применены в таком контексте, прочитайте статью <<On
  Holy Wars and a Plea for Peace>> Дэнни Кохена, доступную по адресу
  \url{http://khavrinen.lcs.mit.edu/wollman/ien-137.txt}.}, когда наиболее значащий байт идёт
первым, мы можем прочитать такое число с помощью следующей функции:

\begin{myverb}
(defun read-u2 (in)
  (+ (* (read-byte in) 256) (read-byte in)))
\end{myverb}

Однако Common Lisp предоставляет более удобный способ осуществления такого рода операций
с битами. Функция \lstinline{LDB}, чьё имя происходит от load byte, может быть использована
для извлечения и присваивания (с помощью \lstinline{SETF}) любого/любому числу идущих подряд
бит из целого числа\footnote{\lstinline{LDB} и связанная функция \lstinline{DPB} были названы по
  функциям ассемблера DEC PDP-10, которые осуществляли в точности эти же вещи. Обе функции
  осуществляют операции над целыми числами, как если бы они хранились в формате дополнения
  до двух, вне зависимости от внутреннего представления, используемого определённой
  реализацией Common Lisp.}. Число бит и их местоположение в целом числе задаются
спецификатором байта, создаваемым функцией \lstinline{BYTE}. \lstinline{BYTE} получает два
аргумента, число бит для извлечения (или присваивания) и позицию самого правого бита, где
наименее значимый бит имеет нулевую позицию. \lstinline{LDB} принимает спецификатор байта и
целое, из которого нужно извлечь биты, и возвращает положительное целое, представляющее
извлечённые биты. Таким образом, мы можем извлечь наименее значащие восемь бит из целого
числа подобным образом:

\begin{myverb}
(ldb (byte 8 0) #xabcd) ==> 205 ; 205 is #xcd
\end{myverb}

Для получения следующих восьми бит нам нужно использовать спецификатор байта
\lstinline{(byte 8 8)} следующим образом:

\begin{myverb}
(ldb (byte 8 8) #xabcd) ==> 171 ; 171 is #xab
\end{myverb}

Мы можем использовать \lstinline{LDB} с \lstinline{SETF} для присваивания заданным битам целого
числа, сохранённого в SETFable месте.

\begin{myverb}
CL-USER> (defvar *num* 0)
*NUM*
CL-USER> (setf (ldb (byte 8 0) *num*) 128)
128
CL-USER> *num*
128
CL-USER> (setf (ldb (byte 8 8) *num*) 255)
255
CL-USER> *num*
65408
\end{myverb}

Итак, мы можем написать \lstinline{read-u2} также следующим образом\footnote{Common Lisp также предоставляет функции для сдвига и маскирования битов целых чисел способом, который может быть более знаком программистам на C и Java. Например, мы можем написать \lstinline{read-u2} третьим способом путём использования этих функций следующим образом:

\begin{myverb}
(defun read-u2 (in)
  (logior (ash (read-byte in) 8) (read-byte in)))
\end{myverb}

\noindent{}что будет примерным эквивалентом следующего метода Java:

\begin{lstlisting}[language=Java]
public int readU2 (InputStream in) throws IOException {
  return (in.read() << 8) | (in.read());
}
\end{lstlisting}

Имена \lstinline{LOGIOR} и \lstinline{ASH} являются сокращениями для \textit{LOGical Inclusive
  OR} и \textit{Arithmetic SHift}. \lstinline{ASH} производит сдвиг целого числа на заданное
количество бит влево, если её второй аргумент является положительным, или вправо, если её
второй аргумент отрицателен. \lstinline{LOGIOR} комбинирует целые путём осуществления
логической операции ИЛИ над каждым их битом. Другая функция, \lstinline{LOGAND}, осуществляет
побитовое И, которое может быть использовано для маскирования определённых битов. Однако
для тех видов операций манипулирования с битами, которые мы будем осуществлять в следующей
главе и далее, \lstinline{LDB} и \lstinline{BYTE} обе будут более удобными и более идиоматичными
для стиля Common Lisp.}:

\begin{myverb}
(defun read-u2 (in)
  (let ((u2 0))
    (setf (ldb (byte 8 8) u2) (read-byte in))
    (setf (ldb (byte 8 0) u2) (read-byte in))
    u2))
\end{myverb}

Для записи числа как 16-битового целого нам нужно извлечь отдельные байты и записать их
один за одним. Для извлечения отдельных байтов нам просто нужно воспользоваться
\lstinline{LDB} с такими же спецификаторами байтов.

\begin{myverb}
(defun write-u2 (out value)
  (write-byte (ldb (byte 8 8) value) out)
  (write-byte (ldb (byte 8 0) value) out))
\end{myverb}

Конечно, мы также можем закодировать целые множеством других способов: с различным числом
байтов, в различном порядке, а также путём использования знакового или беззнакового
формата.

\section{Строки в двоичных файлах}

Текстовые строки являются ещё одним видом примитивных типов данных, который мы можем найти
во многих двоичных форматах. Мы не можем считывать и записывать строки напрямую, читая
файлы побайтно,~--- нам нужно побайтно декодировать и кодировать их, как мы делали это для
двоично кодируемых чисел. И как кодировать целые числа мы можем не одним способом,
кодировать строки мы так же можем множеством способов. Поэтому для начала формат двоичных
чисел должен определять то, как кодируются отдельные знаки.

Для преобразования байт в знаки нам необходимо знать используемые знаковый \textit{код}
(character \textit{code}) и \textit{кодировку} знаков (character
\textit{encoding}). Знаковый код определяет отображение множества положительных целых
чисел на множество знаков. Каждое число отображения называется \textit{единицей
  кодирования} (\textit{code point}). Например, ASCII является знаковым кодом, который
отображает числа интервала 0--127 на знаки, исполь\-зую\-щие\-ся в латинском алфавите. Кодировка
знаков, с другой стороны, определяет, как кодовые единицы представляются в виде
последовательности байт в байт-ориентированной среде, такой как файл. Для кодов, которые
используют восемь или менее бит, таких как ASCII и ISO-8859-1, кодировка тривиальна:
каждое численное значение кодируется едиственным байтом.

Почти так же просты чистые двухбайтовые кодировки, такие как UCS-2, которые осуществляют
отображение между 16-битными значениями и знаками. Единственной причиной, по которой
двухбайтовые кодировки могут оказаться более сложными, чем однобайтовые, является то, что
нам может понадобиться также знать, подразумевается ли кодирование 16-битных значений в
обратном порядке байт либо же в прямом.

Кодировки с переменной длиной используют различное число октетов для различных численных
значений, делая их более сложными, но позволяя им быть более лаконичными в большинстве
случаев. Например, UTF-8, кодировка, спроектированная для использования с кодом знаков
Unicode, использует лишь один октет для кодирования значений из интервала 0--127 и в то же
время до четырёх октетов для кодирования значений до 1,114,111\footnote{Изначально UTF-8
  была спроектирована для представления 31-битного знакового кода и использовала до шести
  байт на единицу кодирования. Однако максимальной единицей кодирования Unicode является
  \lstinline!#x10ffff!, и поэтому Unicode-кодировка UTF-8 требует максимум четырёх байтов на
  единицу кодирования.}.

Так как единицы кодирования из интервала 0--127 отображаются в Unicode на те же знаки, что
и в кодировке ASCII, то закодированный кодировкой UTF-8 текст, состоящий только из знаков
ASCII, будет эквивалентен этому же тексту, но закодированному кодировкой ASCII. С другой
стороны, текст, состоящий преимущественно из знаков, требующих четырёх байт в UTF-8, может
быть более компактно закодирован простой двухбайтовой кодировкой.

Common Lisp предоставляет две функции для преобразования между численными кодами знаков и
объектами знаков: \lstinline{CODE-CHAR}, которая получает численный код и возвращает знак, и
\lstinline{CHAR-CODE}, которая получает знак и возвращает его численный код. Стандарт языка
не определяет, какую кодировку знаков должны использовать реализации языка, поэтому нет
гарантии того, что мы сможем представить любой знак, который может быть закодирован в
данном формате файла, как знак Lisp. Однако почти все современные реализации Common Lisp
используют ASCII, ISO-8859-1 или Unicode в качестве своего внутреннего знакового кода. Так
как Unicode является надмножеством ISO-8859-1, который, в свою очередь, является
надмножеством ASCII, то, если ваша реализация Lisp использует Unicode, \lstinline{CODE-CHAR}
и \lstinline{CHAR-CODE} могут быть использованы напрямую для преобразования любого из этих
трёх знаковых кодов\footnote{Если нам нужно производить разбор формата файлов, который
  использует другие знаковые коды, или делать то же самое для файлов, содержащих
  произвольные строки Unicode, используя не Unicode-реализацию Common Lisp, мы всегда
  можем представить такие строки в памяти как векторы целочисленных единиц
  кодирования. Они не будут строками Lisp, и поэтому мы не сможем манипулировать ими или
  сравнивать их с помощью строковых функций, но мы по-прежнему сможем делать с ними все
  то, что мы можем делать с произвольными векторами.}.

Вдобавок к определению кодирования знаков кодирование строк должно также определять то,
как кодируется длина строк. В двоичных форматах файлов обычно используются три техники.

Простейшая заключается в том, чтобы никак не кодировать длину, которая неявно определяется
по местоположению строки в некоторой большей структуре: некоторый элемент файла может
всегда быть строкой определённой длины, либо строка может быть последним элементом
структуры данных переменной длины, общий размер которой определяет, как много байт осталось
прочитать как данные строки. Оба этих подхода используются в тегах ID3, как мы увидим в
следующей главе.

Другие две техники могут использоваться для кодирования строк переменной длины без
необходимости полагаться на контекст. Одной из них является кодирование длины строки, за
которой следуют данные этой строки,~--- анализатор считывает численное значение (в каком-то
заданном целочисленном формате), а затем считывает это число знаков. Другой техникой
является запись данных строки, за которыми следует разделитель, который не может появиться
внутри строки, такой как нулевой знак (null character).

Различные представления имеют различные преимущества и недостатки, но когда мы имеем дело
с уже заданными двоичными форматами, мы не имеем никакого контроля над тем, какая
кодировка используется. Однако никакая из кодировок не является более сложной для
чтения/записи, чем любая другая. Вот, например, функция, осуществляющая чтение
завершающейся нулевым знаком строки ASCII, под\-ра\-зуме\-ваю\-щая, что ваша реализация Lisp
использует ASCII либо одно из её надмножеств, такое как ISO-8859-1 или Unicode, в качестве
своей внутренней кодировки:

\begin{myverb}
(defconstant +null+ (code-char 0))

(defun read-null-terminated-ascii (in)
  (with-output-to-string (s)
    (loop for char = (code-char (read-byte in))
          until (char= char +null+) do (write-char char s))))
\end{myverb}

Макрос \lstinline{WITH-OUTPUT-TO-STRING}, который упоминался в главе~\ref{ch:14}, является
простым способом построения строки в случае, когда мы не знаем, какой длины она
окажется. Этот макрос создаёт \lstinline{STRING-STREAM} и связывает его с указанным именем
переменной, в данном случае \lstinline{s}. Все знаки, записанные в этот поток, будут собраны в
строку, которая затем будет возвращена в качестве значения формы
\lstinline{WITH-OUTPUT-TO-STRING}.

Для записи строки нам просто нужно преобразовать знаки обратно в численные значения,
которые могут быть записаны с помощью \lstinline{WRITE-BYTE}, а затем записать признак конца
строки после её содержимого.

\begin{myverb}
(defun write-null-terminated-ascii (string out)
  (loop for char across string
        do (write-byte (char-code char) out))
  (write-byte (char-code +null+) out))
\end{myverb}

Как показывает этот пример, главной интеллектуальной задачей (может? и не совсем таковой,
но все же) чтения и записи базовых элементов двоичных файлов является понимание того, как
именно интерпретировать байты файла и отображать их на типы данных Lisp. Если формат
двоичного файла хорошо определён, это может оказаться довольно простой задачей. Фактически
написание функций для чтения и записи данных, закодированных определённым образом,
является просто вопросом программирования.

Теперь мы можем перейти к задаче чтения и записи более сложных структур на диске (on-disk
structures) и отображения их на объекты Lisp.

\section{Составные структуры}

Так как двоичные форматы обычно используются для представления данных способом, который
делает лёгким их отображение на структуры данных в памяти, не должно вызывать удивление
то, что сложные структуры на диске (on-disc structures) обычно определяются схожим
способом с тем, как языки программирования определяют структуры данных в памяти. Обычно
сложные структуры на диске состоят из некоторого числа именованных частей, каждая их
которых является либо примитивным типом, таким как число или строка, либо другой сложной
структурой, либо коллекцией таких значений.

Например, тег ID3, определённый версией 2.2 спецификации, состоит из заголовка, в свою
очередь, состоящего из ISO-8859-1 строки длиной в три знака, которыми всегда являются
<<ID3>>; двух однобайтных беззнаковых целых, которые задают старший номер версии и ревизию
спецификации; восьми бит, являющихся булевыми флагами; и четырёх байт, которые кодируют
размер тега в кодировке, особенной для спецификации ID3. За заголовком идёт список
\textit{фреймов}, каждый из которых имеет свою собственную внутреннюю структуру. За
фреймами идёт столько нулевых байт, сколько необходимо для заполнения тега до размера,
указанного в заголовке.

Если вы глядите на мир через призму объектной ориентации, сложные структуры выглядят
весьма похожими на классы. Например, мы можем написать класс для представления тега ID3.

\begin{myverb}
(defclass id3-tag ()
  ((identifier    :initarg :identifier    :accessor identifier)
   (major-version :initarg :major-version :accessor major-version)
   (revision      :initarg :revision      :accessor revision)
   (flags         :initarg :flags         :accessor flags)
   (size          :initarg :size          :accessor size)
   (frames        :initarg :frames        :accessor frames)))
\end{myverb}

Экземпляр этого класса может быть отличным местом хранения данных, необходимых для
представления тега ID3. Затем мы можем написать функции чтения и записи экземпляров этого
класса. Например, предположив существование функций чтения соответствующих примитивных
типов данных, функция \lstinline{read-id3-tag} может выглядеть следующим образом:

\begin{myverb}
(defun read-id3-tag (in)
  (let ((tag (make-instance 'id3-tag)))
    (with-slots (identifier major-version revision flags size frames) tag
      (setf identifier    (read-iso-8859-1-string in :length 3))
      (setf major-version (read-u1 in))
      (setf revision      (read-u1 in))
      (setf flags         (read-u1 in))
      (setf size          (read-id3-encoded-size in))
      (setf frames        (read-id3-frames in :tag-size size)))
    tag))
\end{myverb}

Функция \lstinline{write-id3-tag} будет структурирована схожим образом: мы будем использовать
соответствующие функции \lstinline{write-*} для записи значений, хранящихся в слотах объекта
\lstinline{id3-tag}.

Несложно увидеть, как мы можем написать соответствующие классы для представления всех
сложных структур данных спецификации наряду с функциями \lstinline{read-foo} и \lstinline{write-foo}
для каждого класса и необходимых примитивных типов. Но также легко заметить, что все
функции чтения и записи будут весьма похожими, отличающимися только тем, данные каких
типов они читают, и именами слотов, в которые они сохраняют эти данные. Это станет
особенно утомительным, когда мы учтём тот факт, что описание структуры тега ID3 заняло
почти четыре строки текста, в то время как мы уже написали одиннадцать строк кода, все ещё
не написав \lstinline{write-id3-tag}.

Что нам действительно нужно, так это способ описания структур наподобие тега ID3 в форме,
которая лаконична так же, как и псевдокод спецификации, и чтобы это описание раскрывалось
в код, который определяет класс \lstinline{id3-tag} и функции, осуществляющие преобразование
между байтами на диске и экземплярами этого класса. Звучит как работа для системы
макросов.

\section{Проектирование макросов}

Так как мы уже имеем примерное представление о том, какой код ваш макрос должен
генерировать, следующим шагом в соответствии с процессом написания макросов, описанным
мною в главе~\ref{ch:08}, является смена ракурса и размышления о том, как должен выглядеть
вызов этого макроса. Так как целью является иметь возможность написания чего-то, столь же
краткого, как и псевдокод спецификации ID3, мы можем начать с него. Заголовок тега ID3
определяется следующим образом:

\begin{myverb}
ID3/file identifier      "ID3"
ID3 version              \$02 00
ID3 flags                %xx000000
ID3 size             4 * %0xxxxxxx
\end{myverb}

В нотации спецификации это означает, что слот <<file identifier>> тега ID3 является строкой
<<ID3>> в кодировке ISO-8859-1. Слот version состоит из двух байт, первый из которых, для
данной версии спецификации, имеет значение 2 и второй, опять же для данной версии,~---
0. Слот flags имеет размер в восемь бит, все из которых, кроме первых двух, имеют нулевое
значение, а size состоит из четырёх байт, каждый из которых содержит 0 в своём старшем
разряде.

Некоторая часть информации не охватывается этим псевдокодом. Например? то, как именно
интерпретируются четыре байта, кодирующие размер, описывается несколькими строками
текста. Схожим образом спецификация описывает текстом то, как после заголовка сохраняются
фрейм и последующие байты заполнения. Но всё же большая часть того, что нам нужно знать,
для того чтобы написать код чтения и записи тега ID3, задаётся этим псевдокодом. Таким
образом, мы должны иметь возможность напи\-сания варианта этого псевдокода s-выражением,
которое раскроется в класс, и определения функций, которые нам иначе бы пришлось писать
вручную: что-то, возможно, вроде этого:

\begin{myverb}
(define-binary-class id3-tag
  ((file-identifier (iso-8859-1-string :length 3))
   (major-version   u1)
   (revision        u1)
   (flags           u1)
   (size            id3-tag-size)
   (frames          (id3-frames :tag-size size))))
\end{myverb}

Основной идеей является то, что эта форма определяет класс \lstinline{id3-tag} подобно тому,
как мы можем сделать сами с помощью \lstinline{DEFCLASS}, но вместо определения таких вещей,
как \lstinline{:initarg} и \lstinline{:accessor}, каждое определение слота состоит из имени
слота~--- \lstinline{file-identifier}, \lstinline{major-version} и т.~д.~--- и информации о том, как
этот слот представляется на диске. Так как мы всего лишь немного пофантазировали, нам не
нужно беспокоиться о том, как именно макрос \lstinline{define-binary-class} будет знать, что
делать с такими выражениями, как \lstinline{(iso-8859-1-string :length 3)}, \lstinline{u1},
\lstinline{id3-tag-size} и \lstinline{(id3-frames :tag-size size)}; пока каждое выражение содержит
информацию, необходимую для знания того, как читать и записывать определённые данные, все
должно быть хорошо.

\section{Делаем мечту реальностью}

Хорошо, достаточно фантазий о хорошо выглядящем коде; теперь нужно приступать к работе по
написанию \lstinline{define-binary-class}: написанию кода, который будет преобразовывать
краткое выражение, описывающее, как выглядит тег ID3, в код, который может представлять
этот тег в памяти, считывать с диска и записывать его обратно.

Для начала нам стоит определить пакет для нашей библиотеки. Вот файл пакета, который
поставляется с версией, которую вы можете скачать с веб-сайта книги:% TODO: add url

\begin{myverb}
(in-package :cl-user)

(defpackage :com.gigamonkeys.binary-data
  (:use :common-lisp :com.gigamonkeys.macro-utilities)
  (:export :define-binary-class
           :define-tagged-binary-class
           :define-binary-type
           :read-value
           :write-value
           :*in-progress-objects*
           :parent-of-type
           :current-binary-object
           :+null+))
\end{myverb}

Пакет \lstinline{COM.GIGAMONKEYS.MACRO-UTILITIES} содержит макросы \lstinline{with-gensyms} и
\lstinline{once-only} из главы~\ref{ch:08}.

Так как мы уже имеем написанную вручную версию кода, который хотим сгенерировать, не
должно быть очень сложно написать такой макрос. Просто разберём его на небольшие части,
начав с версии \lstinline{define-binary-class}, которая просто генерирует форму
\lstinline{DEFCLASS}.

Если мы вновь взглянем на форму \lstinline{define-binary-class}, то увидим, что она принимает
два аргумента: имя \lstinline{id3-tag} и список спецификаторов слотов, каждый из которых сам
является двухэлементным списком. По этим частям нам нужно построить соответствующую форму
\lstinline{DEFCLASS}. Очевидно, что наибольшее различие между формой
\lstinline{define-binary-class} и правильной формой \lstinline{DEFCLASS} заключается в
спецификаторах слотов. Одиночный спецификатор слота из \lstinline{define-binary-class} выглядит
подобным образом:

\begin{myverb}
(major-version u1)
\end{myverb}

Но это не является верным спецификатором слота для \lstinline{DEFCLASS}. Вместо этого нам
нужно что-то вот такое:

\begin{myverb}
(major-version :initarg :major-version :accessor major-version)
\end{myverb}

Достаточно просто. Для начала определим простую функцию преобразования символа в
соответствующий ключевой символ.

\begin{myverb}
(defun as-keyword (sym) (intern (string sym) :keyword))
\end{myverb}

Теперь определим функцию, которая получает спецификатор слота \lstinline{define-binary-class} и возвращает спецификатор слота \lstinline{DEFCLASS}.

\begin{myverb}
(defun slot->defclass-slot (spec)
  (let ((name (first spec)))
    `(,name :initarg ,(as-keyword name) :accessor ,name)))
\end{myverb}

Мы можем протестировать эту функцию в REPL после переключения в наш новый пакет путём
вызова \lstinline{IN-PACKAGE}.

\begin{myverb}
BINARY-DATA> (slot->defclass-slot '(major-version u1))
(MAJOR-VERSION :INITARG :MAJOR-VERSION :ACCESSOR MAJOR-VERSION)
\end{myverb}

Выглядит неплохо. Теперь написание первой версии \lstinline{define-binary-class} тривиально.

\begin{myverb}
(defmacro define-binary-class (name slots)
  `(defclass ,name ()
     ,(mapcar #'slot->defclass-slot slots)))
\end{myverb}

Это простой макрос, написанный в template-стиле: \lstinline{define-binary-class} генерирует
форму \lstinline{DEFCLASS} путём подстановки (interpolating) имени класса и списка
спецификаторов слотов, сконструированного путём применения \lstinline{slot->defclass-slot} к
каждому элементу списка спецификаторов слотов формы \lstinline{define-binary-class}.

Для просмотра кода, который генерирует этот макрос, мы можем вычислить в REPL следующее
выражение:

\begin{myverb}
(macroexpand-1 '(define-binary-class id3-tag
  ((identifier      (iso-8859-1-string :length 3))
   (major-version   u1)
   (revision        u1)
   (flags           u1)
   (size            id3-tag-size)
   (frames          (id3-frames :tag-size size)))))
\end{myverb}

Результат, слегка переформатированный в целях улучшения читаемости, должен казаться вам
знакомым, так как это в точности то определение класса, которое мы написали вручную ранее:

\begin{myverb}
(defclass id3-tag ()
  ((identifier      :initarg :identifier    :accessor identifier)
   (major-version   :initarg :major-version :accessor major-version)
   (revision        :initarg :revision      :accessor revision)
   (flags           :initarg :flags         :accessor flags)
   (size            :initarg :size          :accessor size)
   (frames          :initarg :frames        :accessor frames)))
\end{myverb}

\section{Чтение двоичных объектов}

Следующим шагом нам нужно заставить \lstinline{define-binary-class} также генерировать функцию,
которая может прочитать экземпляр нового класса. Учитывая функцию \lstinline{read-id3-tag},
написанную нами ранее, кажется, это будет немного сложнее, так как \lstinline{read-id3-tag} не
является столь же однородной: для чтения значений каждого слота нам приходится вызывать
различные функции, не говоря уже о том, что имя функции, \lstinline{read-id3-tag}, хоть и
получается из имени определяемого нами класса, не является одним из аргументов
\lstinline{define-binary-class}, а следовательно, не может быть просто подставлено в шаблон.

Мы можем решить обе эти проблемы, следуя такому соглашению по именованию, при котором
макрос сможет вычислять имя функции, основываясь на имени типа в спецификаторе
слота. Однако тогда \lstinline{define-binary-class} придётся генерировать имя
\lstinline{read-id3-tag}, что возможно, но является плохой идеей. Макросам, создающим
глобальные определения, следует в общем случае использовать только имена, переданные им;
макросы, сами генерирующие имена, могут привести к сложно предсказуемым и
трудно отлаживаемым конфликтам имён, когда сгенерированные имена оказываются теми же, что
уже где-нибудь используются\footnote{К сожалению, сам язык не всегда подаёт хороший пример
  в этом отношении: макрос \lstinline{DEFSTRUCT}, которого я не обсуждал, так как он почти
  полностью вытеснен \lstinline{DEFCLASS}, генерирует функции с именами, получающимися на
  основе имени, данного структуре. Плохой пример \lstinline{DEFSTRUCT} сбивает с истинного
  пути многих новичков.}.

Мы можем избежать обоих этих неудобств, заметив, что все функции, считывающие значения
определённого типа, имеют в своей сути одинаковую цель: считывание значения определённого
типа из потока. Говоря просто, мы можем увидеть, что все они являются экземплярами одной
обобщённой операции. И простое использование слова <<обобщённый>> должно подтолкнуть вас
прямо к решению проблемы: вместо определения множества независимых функций, имеющих
различные имена, мы можем определить одну обобщённую функцию \lstinline{read-value} с методами,
специализированными для чтения значений различных типов.

Таким образом, вместо определения функций \lstinline{read-iso-8859-1-string} и \lstinline{read-u1}
мы можем определить \lstinline{read-value} как обобщённую функцию, принимающую два обязательных
аргумента: тип и поток, а также, возможно, некоторые ключевые аргументы.

\begin{myverb}
(defgeneric read-value (type stream &key)
  (:documentation "Read a value of the given type from the stream."))
\end{myverb}

Путём указания \lstinline!&key! без самих ключевых параметров мы позволяем различным
методам определять свои собственные \lstinline!&key! параметры, но не требуя этого от
них. Это значит, что каждый метод, специализирующий \lstinline{read-value}, должен будет
включить либо \lstinline!&key!, либо \lstinline!&rest! в свой список параметров, чтобы
быть совместимым с обобщённой функцией.

Затем мы определяем методы, использующие специализаторы EQL для специализации аргумента
типа по имени типа значений, которые хотим считывать.

\begin{myverb}
(defmethod read-value ((type (eql 'iso-8859-1-string)) in &key length) ...)

(defmethod read-value ((type (eql 'u1)) in &key) ...)
\end{myverb}

Затем мы можем изменить \lstinline{define-binary-class} так, чтобы он генерировал метод
\lstinline{read-value}, специализированный по имени типа \lstinline{id3-tag} и реализованный в
терминах вызовов \lstinline{read-value} с соответствующими типами слотов в качестве первого
аргумента. Код, который мы хотим сгенерировать, выглядит следующим образом:

\begin{myverb}
(defmethod read-value ((type (eql 'id3-tag)) in &key)
  (let ((object (make-instance 'id3-tag)))
    (with-slots (identifier major-version revision flags size frames) object
      (setf identifier    (read-value 'iso-8859-1-string in :length 3))
      (setf major-version (read-value 'u1 in))
      (setf revision      (read-value 'u1 in))
      (setf flags         (read-value 'u1 in))
      (setf size          (read-value 'id3-encoded-size in))
      (setf frames        (read-value 'id3-frames in :tag-size size)))
    object))
\end{myverb}

Теперь, так же как для генерации формы \lstinline{DEFCLASS}, нам нужна была функция,
транслирующая спецификатор слота \lstinline{define-binary-class} в спецификатор слота
\lstinline{DEFCLASS}, теперь нам нужна функция, получающая спецификатор слота
\lstinline{define-binary-class} и генерирующая соответствующую форму \lstinline{SETF}, то есть
что-то, получающее вот такое:

\begin{myverb}
(identifier (iso-8859-1-string :length 3))
\end{myverb}

\noindent{}и возвращающее это:

\begin{myverb}
(setf identifier (read-value 'iso-8859-1-string in :length 3))
\end{myverb}

Однако существует различие между этим кодом и спецификатором слота \lstinline{DEFCLASS}:
этот код включает в себя ссылку на переменную \lstinline{in}, параметр метода
\lstinline{read-value}, который не был получен из спецификатора слота. Он не обязательно должен
называться \lstinline{in}, но какое бы имя мы не использовали, оно должно быть тем же, что
используется в списке параметров метода, а также в других вызовах
\lstinline{read-value}. Сейчас мы можем уклониться от проблемы того, откуда получается это имя,
определив \lstinline{slot->read-value} таким образом, чтобы она принимала второй аргумент,
содержащий имя переменной потока.

\begin{myverb}
(defun slot->read-value (spec stream)
  (destructuring-bind (name (type &rest args)) (normalize-slot-spec spec)
    `(setf ,name (read-value ',type ,stream ,@args))))
\end{myverb}

Функция \lstinline{normalize-slot-spec} нормализует второй элемент спецификатора слота, преобразуя символ, такой как \lstinline{u1}, в список \lstinline{(u1)}, так что \lstinline{DESTRUCTURING-BIND} может осуществить его разбор. Она выглядит так:

\begin{myverb}
(defun normalize-slot-spec (spec)
  (list (first spec) (mklist (second spec))))

(defun mklist (x) (if (listp x) x (list x)))
\end{myverb}

Мы можем протестировать \lstinline{slot->read-value} с каждым типом спецификаторов слотов.

\begin{myverb}
BINARY-DATA> (slot->read-value '(major-version u1) 'stream)
(SETF MAJOR-VERSION (READ-VALUE 'U1 STREAM))
BINARY-DATA> (slot->read-value '(identifier (iso-8859-1-string :length 3)) 'stream)
(SETF IDENTIFIER (READ-VALUE 'ISO-8859-1-STRING STREAM :LENGTH 3))
\end{myverb}

Со всеми этими функциями мы уже готовы добавить \lstinline{read-value} в
\lstinline{define-binary-class}. Если мы возьмём вручную написанный метод \lstinline{read-value} и
удалим из него все то, что касается определённого класса, у нас останется следующий
каркас:

\begin{myverb}
(defmethod read-value ((type (eql ...)) stream &key)
  (let ((object (make-instance ...)))
    (with-slots (...) object
      ...
    object)))
\end{myverb}

Все, что нам нужно сделать,~--- это добавить этот каркас в шаблон \lstinline{define-binary-class},
заменив многоточия кодом, который заполнит этот каркас подходящими именами и кодом. Мы
также захотим заменить переменные \lstinline{type}, \lstinline{stream} и \lstinline{object}
сгенерированными \lstinline{GENSYM} именами для избежания потенциальных конфликтов с именами
слотов\footnote{Технически для \lstinline{type} или \lstinline{object} не существует возможности
  конфликтования с именами слотов: в худшем случае они будут скрыты внутри формы
  \lstinline{WITH-SLOTS}. Но все же не будет ничего плохого в том, чтобы просто
  сгенерировать с помощью \lstinline{GENSYM} все локальные переменные, используемые внутри
  шаблона макроса.}, что мы можем сделать с помощью макроса \lstinline{with-gensyms},
рассмотренного в главе~\ref{ch:08}.

Также, так как макрос должен раскрываться в одиночную форму, мы должны <<обернуть>>
какую-то вокруг \lstinline{DEFCLASS} и \lstinline{DEFMETHOD}. Обычно для макросов, которые
раскрываются в несколько определений, используется \lstinline{PROGN} из-за спе\-циаль\-ной
трактовки, которую она получает от компилятора, когда находится на верхнем уровне файла,
что было обсуждено в главе~\ref{ch:20}.

Таким образом, мы можем изменить \lstinline{define-binary-class} следующим образом:

\begin{myverb}
(defmacro define-binary-class (name slots)
  (with-gensyms (typevar objectvar streamvar)
    `(progn
       (defclass ,name ()
         ,(mapcar #'slot->defclass-slot slots))

       (defmethod read-value ((,typevar (eql ',name)) ,streamvar &key)
         (let ((,objectvar (make-instance ',name)))
           (with-slots ,(mapcar #'first slots) ,objectvar
             ,@(mapcar #'(lambda (x) (slot->read-value x streamvar)) slots))
           ,objectvar)))))
\end{myverb}

\section{Запись двоичных объектов}

Генерация кода для записи экземпляра двоичного класса происходит схожим образом. Для
начала мы можем определить обобщённую функцию \lstinline{write-value}.

\begin{myverb}
(defgeneric write-value (type stream value &key)
  (:documentation "Write a value as the given type to the stream."))
\end{myverb}

Затем мы определяем вспомогательную функцию, которая транслирует спецификатор слота
\lstinline{define-binary-class} в код, который записывает этот слот с помощью
\lstinline{write-value}. Как и для функции \lstinline{slot->read-value}, эта вспомогательная функция
принимает имя переменной потока в качестве параметра.

\begin{myverb}
(defun slot->write-value (spec stream)
  (destructuring-bind (name (type &rest args)) (normalize-slot-spec spec)
    `(write-value ',type ,stream ,name ,@args)))
\end{myverb}

После этого мы можем добавить шаблон \lstinline{write-value} в макрос
\lstinline{define-binary-class}.

\begin{myverb}
(defmacro define-binary-class (name slots)
  (with-gensyms (typevar objectvar streamvar)
    `(progn
       (defclass ,name ()
         ,(mapcar #'slot->defclass-slot slots))

       (defmethod read-value ((,typevar (eql ',name)) ,streamvar &key)
         (let ((,objectvar (make-instance ',name)))
           (with-slots ,(mapcar #'first slots) ,objectvar
             ,@(mapcar #'(lambda (x) (slot->read-value x streamvar)) slots))
           ,objectvar))

       (defmethod write-value ((,typevar (eql ',name)) ,streamvar ,objectvar &key)
         (with-slots ,(mapcar #'first slots) ,objectvar
           ,@(mapcar #'(lambda (x) (slot->write-value x streamvar)) slots))))))
\end{myverb}

\section{Добавление наследования и помеченных (tagged) структур}

Хотя эта версия \lstinline{define-binary-class} будет обрабатывать автономные (stand-alone)
структуры, двоичные форматы файлов часто определяют такие структуры на диске (on-disk
structures), которые было бы естественно моделировать отношениями подклассов и
суперклассов. Поэтому мы можем захотеть расширить \lstinline{define-binary-class} для поддержки
наследования.

Родственной техникой, используемой во многих двоичных форматах, является такая, при
которой имеется множество структур данных на диске, точный тип которых может быть
определён только путём считывания некоторых данных, которые указывают, как осуществлять
разбор последующих байтов. Например, фреймы, которые составляют большую часть тега ID3,
все разделяют общую структуру заголовка, состоящую из строкового идентификатора и
длины. Для чтения фрейма нам нужно прочитать идентификатор и использовать его значение для
определения вида просматриваемого фрейма, а следовательно, того, как осуществлять разбор
его тела.

Текущая версия макроса \lstinline{define-binary-class} не предоставляет способа осуществления
такого рода считывания: мы можем использовать \lstinline{define-binary-class} для определения
класса, представляющего любой вид фрейма, но мы не имеем возможности узнать, какой тип
фрейма считывать, без считывания, по меньшей мере, идентификатора. И если другой код
считывает идентификатор, чтобы определить, какой тип передавать функции \lstinline{read-value},
то это нарушит работу \lstinline{read-value}, поскольку она ожидает возможности считать все
данные, составляющие экземпляр класса, создаваемый ею.

Мы можем решить эту проблему, добавив возможность наследования в
\lstinline{define-binary-class}, а затем написав другой макрос
\lstinline{define-tagged-binary-class}, предназначенный для определения <<абстрактных>>
классов, экземляры которых не создаются напрямую, но по которым могут быть
специализированы методы \lstinline{read-value}, которые знают, как считывать достаточно данных
для определения конкретного класса, экземпляр которого нужно создать.

Первым шагом добавления возможности наследования в \lstinline{define-binary-class} является
добавление в макрос параметра, принимающего список суперклассов.

\begin{myverb}
(defmacro define-binary-class (name (&rest superclasses) slots) ...
\end{myverb}

Затем в шаблоне \lstinline{DEFCLASS} подставим это значение вместо пустого списка.

\begin{myverb}
(defclass ,name ,superclasses
  ...)
\end{myverb}

Однако нужно сделать немного больше. Нам также нужно изменить методы \lstinline{read-value} и
\lstinline{write-value} так, чтобы методы, сгенерированные при определении супер\-клас\-са, могли
бы быть использованы методами, сгенерированными как часть подкласса, для чтения и записи
наследуемых слотов.

Способ, которым работает текущая версия \lstinline{read-value}, совершенно не подходит, так как
он инстанцирует объект перед его заполнением. То есть у вас есть метод, ответственный за
чтение полей суперкласса, инстанцирующий один объект, и метод подкласса, инстанцирующий и
заполняющий другой объект.

Мы можем исправить эту проблему путём разделения \lstinline{read-value} на две части: одну~---
ответственную за инстанцирование правильного вида объекта, а другую~--- за заполнение
слотов уже существующего объекта. На стороне записи все несколько проще, но и там мы можем
использовать схожую технику.

Поэтому мы определим две новые обобщённые функции: \lstinline{read-object} и
\lstinline{write-object}, обе получающие существующий объект и поток. Методы этих обобщённых
функций будут ответственны за чтение и запись слотов, специфичных для классов, для которых
они специализированы.

\begin{myverb}
(defgeneric read-object (object stream)
  (:method-combination progn :most-specific-last)
  (:documentation "Fill in the slots of object from stream."))

(defgeneric write-object (object stream)
  (:method-combination progn :most-specific-last)
  (:documentation "Write out the slots of object to the stream."))
\end{myverb}

Определение этих обобщённых функций с использованием комбинатора методов \lstinline{PROGN} с
опцией \lstinline{:most-specific-last} позволяет нам определять методы, специализированные для
каждого двоичного класса и работающие только со слотами, действительно определёнными в
таком классе; комбинатор методов \lstinline{PROGN} скомбинирует все применимые методы так,
что метод, специализированный для наименее специфичного класса в иерархии, выполнится
первым, считывая или записывая слоты, определённые в этом классе, затем выполнится метод,
специализированный для следующего наименее специфичного класса и т.~д. И, так как теперь
вся тяжёлая, специфичная для класса работа осуществляется методами \lstinline{read-object} и
\lstinline{write-object}, нам даже не нужно определять специализированные методы
\lstinline{read-value} и \lstinline{write-value}: мы можем определить методы по умолчанию, которые
считают аргумент типа именем двоичного класса.

\begin{myverb}
(defmethod read-value ((type symbol) stream &key)
  (let ((object (make-instance type)))
    (read-object object stream)
    object))

(defmethod write-value ((type symbol) stream value &key)
  (assert (typep value type))
  (write-object value stream))
\end{myverb}

Обратите внимание на то, как мы можем использовать \lstinline{MAKE-INSTANCE} в ка\-чест\-ве
обобщённой фабрики объектов (generic object factory): хотя обычно мы вызываем
\lstinline{MAKE-INSTANCE} с закавыченным (quoted) символом в качестве первого аргумента, так
как чаще всего знаем, экземпляр какого именно класса хотим создать, мы можем использовать
любое выражение, которое вычисляется в имя класса, как в данном случае используем параметр
\lstinline{type} метода \lstinline{read-value}.

Действительные изменения, внесённые в \lstinline{define-binary-class} для определения методов
\lstinline{read-object} и \lstinline{write-object} вместо \lstinline{read-value} и \lstinline{write-value},
довольно незначительны.

\begin{myverb}
(defmacro define-binary-class (name (&rest superclasses) slots)
  (with-gensyms (objectvar streamvar)
    `(progn
       (defclass ,name ,superclasses
         ,(mapcar #'slot->defclass-slot slots))

       (defmethod read-object progn ((,objectvar ,name) ,streamvar)
         (with-slots ,(mapcar #'first slots) ,objectvar
           ,@(mapcar #'(lambda (x) (slot->read-value x streamvar)) slots)))

       (defmethod write-object progn ((,objectvar ,name) ,streamvar)
         (with-slots ,(mapcar #'first slots) ,objectvar
           ,@(mapcar #'(lambda (x) (slot->write-value x streamvar)) slots))))))
\end{myverb}

\section{Отслеживание унаследованных слотов}

Это определение будет работать для многих случаев. Однако оно не обрабатывает одну
достаточно распространённую ситуацию, а именно когда нам нужен подкласс, которому
необходимо ссылаться на унаследованные слоты в своих собственных определениях
слотов. Например, имея текущее определение \lstinline{define-binary-class}, мы можем определить
следующий одиночный класс:

\begin{myverb}
(define-binary-class generic-frame ()
  ((id (iso-8859-1-string :length 3))
   (size u3)
   (data (raw-bytes :bytes size))))
\end{myverb}

Ссылка на \lstinline{size} в определении \lstinline{data} работает ожидаемым образом, так как
выражения, считывающие и записывающие слот \lstinline{data}, обернуты формой
\lstinline{WITH-SLOTS}, которая перечисляет все слоты объекта. Однако если мы попытаемся
разделить этот класс на два класса следующим образом:

\begin{myverb}
(define-binary-class frame ()
  ((id (iso-8859-1-string :length 3))
   (size u3)))

(define-binary-class generic-frame (frame)
  ((data (raw-bytes :bytes size))))
\end{myverb}

\noindent{}мы получим предупреждение времени компиляции при компиляции определения
\lstinline{generic-frame} и ошибку времени выполнения при попытке его использования, так как в
методах \lstinline{read-object} и \lstinline{write-object}, специализированных для
\lstinline{generic-frame}, не будет лексически видимой переменной \lstinline{size}.

Что нам нужно, так это отслеживание слотов, определённых каждый двоичным классом, а затем
включение всех наследуемых слотов в формы \lstinline{WITH-SLOTS} методов \lstinline{read-object} и
\lstinline{write-object}.

Наиболее простым способом отслеживания подобной информации является связывание её с
символом, именующим класс. Как мы обсуждали в главе~\ref{ch:21}, каждый символьный объект
имеет ассоциированный с ним список свойств, доступ к которому можно получить с помощью
функций \lstinline{SYMBOL-PLIST} и \lstinline{GET}. Мы можем связать произвольную пару
ключ/значение с символом, добавив их в список свойств этого символа с помощью вызова
\lstinline{SETF} для результата \lstinline{GET}. Например, если двоичный класс \lstinline{foo}
определяет три слота \lstinline{x}, \lstinline{y} и \lstinline{z}, мы можем отследить этот факт, добавив
в список свойств символа \lstinline{foo} ключ \lstinline{slots} со значением \lstinline{(x y z)} с
помощью следующего выражения:

\begin{myverb}
(setf (get 'foo 'slots) '(x y z))
\end{myverb}

Мы хотим осуществлять этот учёт как часть вычисления \lstinline{define-binary-class} для
\lstinline{foo}. Однако не совсем очевидно, куда поместить это выражение. Если мы будем
вычислять его при вычислении раскрытий макросов, это выражение вычислится при компиляции
формы \lstinline{define-binary-class}, но не во время последующей загрузки файла, содержащего
полученный скомпилированный код. С другой стороны, если мы включим это выражение в
раскрытие макроса, то оно не будет вычисляться во время компиляции, а это означает, что
при компиляции файла с несколькими формами \lstinline{define-binary-class} никакой информации о
том, какие классы определяют какие слоты, не будет доступно до полной загрузки файла, что
слишком поздно.

Это как раз тот случай, для которого предназначен специальный оператор \lstinline{EVAL-WHEN},
который мы обсуждали в главе~\ref{ch:20}. Обернув форму в \lstinline{EVAL-WHEN}, мы можем
контролировать то, вычисляется ли она во время компиляции, либо во время загрузки
скомпилированного кода, либо в обоих случаях. Для таких случаев, как данный, когда мы
хотим собрать некоторую информацию во время компиляции формы макроса, к которой мы хотим
также иметь доступ после загрузки скомпилированной формы, нам следует обернуть выражения
сбора этой информации в \lstinline{EVAL-WHEN} следующим образом:

\begin{myverb}
(eval-when (:compile-toplevel :load-toplevel :execute)
  (setf (get 'foo 'slots) '(x y z)))
\end{myverb}

\noindent{}и включить форму \lstinline{EVAL-WHEN} в раскрытие, генерируемое макросом. Итак, мы можем
сохранить слоты и прямые суперклассы двоичного класса, добавив следующую форму в раскрытие,
генерируемое \lstinline{define-binary-class}:

\begin{myverb}
(eval-when (:compile-toplevel :load-toplevel :execute)
  (setf (get ',name 'slots) ',(mapcar #'first slots))
  (setf (get ',name 'superclasses) ',superclasses))
\end{myverb}

Теперь мы можем определить три вспомогательные функции для осуществления доступа к этой
информации. Первая просто возвращает слоты, определённые двоичным классом. Хорошей идеей
является возвращение копии списка, так как мы не хотим, чтобы посторонний код
модифицировал список слотов после того, как двоичный класс уже был определён.

\begin{myverb}
(defun direct-slots (name)
  (copy-list (get name 'slots)))
\end{myverb}

Следующая функция возвращает слоты, унаследованные от других двоичных классов.

\begin{myverb}
(defun inherited-slots (name)
  (loop for super in (get name 'superclasses)
        nconc (direct-slots super)
        nconc (inherited-slots super)))
\end{myverb}

И наконец, мы можем определить функцию, которая возвращает список, содержащий имена всех
определённых и унаследованных слотов.

\begin{myverb}
(defun all-slots (name)
  (nconc (direct-slots name) (inherited-slots name)))
\end{myverb}

Итак, мы хотим, чтобы при вычислении раскрытия формы \lstinline{define-binary-class}
генерировалась форма \lstinline{WITH-SLOTS}, содержащая имена всех слотов, определённых в
новом классе и во всех его суперклассах. Однако мы не можем использовать \lstinline{all-slots}
во время генерации раскрытия, так как информация не будет доступна до того момента, когда
это раскрытие будет скомпилировано. Вместо этого нам следует воспользоваться следующей
функцией, получающей список спецификаторов слотов и список суперклассов, переданных в
\lstinline{define-binary-class}, и использующую их для вычисления списка всех слотов нового
класса:

\begin{myverb}
(defun new-class-all-slots (slots superclasses)
  (nconc (mapcan #'all-slots superclasses) (mapcar #'first slots)))
\end{myverb}

Имея эти функции, мы можем изменить \lstinline{define-binary-class} таким образом, чтобы он
сохранял информацию об определяемом в данный момент классе и использовал уже сохранённую
информацию о слотах суперклассов для генерации форм \lstinline{WITH-SLOTS}.

\begin{myverb}
(defmacro define-binary-class (name (&rest superclasses) slots)
  (with-gensyms (objectvar streamvar)
    `(progn
       (eval-when (:compile-toplevel :load-toplevel :execute)
         (setf (get ',name 'slots) ',(mapcar #'first slots))
         (setf (get ',name 'superclasses) ',superclasses))

       (defclass ,name ,superclasses
         ,(mapcar #'slot->defclass-slot slots))

       (defmethod read-object progn ((,objectvar ,name) ,streamvar)
         (with-slots ,(new-class-all-slots slots superclasses) ,objectvar
           ,@(mapcar #'(lambda (x) (slot->read-value x streamvar)) slots)))

       (defmethod write-object progn ((,objectvar ,name) ,streamvar)
         (with-slots ,(new-class-all-slots slots superclasses) ,objectvar
           ,@(mapcar #'(lambda (x) (slot->write-value x streamvar)) slots))))))
\end{myverb}

\section{Помеченные структуры}

Имея возможность определения двоичных классов, которые расширяют другие двоичные классы,
мы готовы определить новый макрос, предназначенный для определения классов, представляющих
<<помеченные>> структуры. В основе нашего способа чтения помеченных структур будет
определение специализированного метода \lstinline{read-value}, который знает, как считывать
значения, составляющие начало структуры, и как затем использовать эти значения для
определения подкласса, экземпляр которого нужно создать. Затем этот метод создаёт
экземпляр класса с помощью \lstinline{MAKE-INSTANCE}, передавая уже прочитанные значения в
качестве аргументов инициализации (initags), и передаёт объект в \lstinline{read-object},
позволяя действительному классу объекта определять, как именно считывается остальная часть
структуры.

Новый макрос \lstinline{define-tagged-binary-class} будет выглядеть как
\lstinline{define-binary-class} с добавочной опцией \lstinline{:dispatch}, используемой для указания
формы, которая должна вычисляться в имя двоичного класса. Форма \lstinline{:dispatch} будет
вычислена в том контексте, в котором имена слотов, определённых помеченным классом,
связываются с переменными, содержащими значения, прочитанные из файла. Класс, чьё имя
возвращается этой формой, должен принимать аргументы инициализации (initargs),
соответствующие именам слотов, определённых помеченным классом. Это легко обеспечивается в
том случае, если форма \lstinline{:dispatch} всегда вычисляется в имя класса, являющегося
подклассом помеченного класса.

Например, представив, что мы имеем функцию \lstinline{find-frame-class}, которая отображает
строковый идентификатор на двоичный класс, представляющий определённый вид фрейма ID3, мы
можем определить помеченный двоичный класс \lstinline{id3-frame} следующим образом:

\begin{myverb}
(define-tagged-binary-class id3-frame ()
  ((id   (iso-8859-1-string :length 3))
   (size u3))
  (:dispatch (find-frame-class id)))
\end{myverb}

Раскрытие \lstinline{define-tagged-binary-class} будет содержать \lstinline{DEFCLASS} и метод \lstinline{write-object} так же, как и раскрытие \lstinline{define-binary-class}, но вместо метода \lstinline{read-object} оно будет содержать метод \lstinline{read-value}, выглядящий следующим образом:

\begin{myverb}
(defmethod read-value ((type (eql 'id3-frame)) stream &key)
  (let ((id (read-value 'iso-8859-1-string stream :length 3))
        (size (read-value 'u3 stream)))
    (let ((object (make-instance (find-frame-class id) :id id :size size)))
      (read-object object stream)
      object)))
\end{myverb}

Так как раскрытия \lstinline{define-tagged-binary-class} и \lstinline{define-binary-class} будут
идентичными, за исключением метода чтения, мы можем вынести общие части во вспомогательный
макрос \lstinline{define-generic-binary-class}, который принимает метод чтения в качестве
параметра и подставляет его в своё раскрытие.

\begin{myverb}
(defmacro define-generic-binary-class (name (&rest superclasses) slots read-method)
  (with-gensyms (objectvar streamvar)
    `(progn
       (eval-when (:compile-toplevel :load-toplevel :execute)
         (setf (get ',name 'slots) ',(mapcar #'first slots))
         (setf (get ',name 'superclasses) ',superclasses))

       (defclass ,name ,superclasses
         ,(mapcar #'slot->defclass-slot slots))

       ,read-method

       (defmethod write-object progn ((,objectvar ,name) ,streamvar)
         (declare (ignorable ,streamvar))
         (with-slots ,(new-class-all-slots slots superclasses) ,objectvar
           ,@(mapcar #'(lambda (x) (slot->write-value x streamvar)) slots))))))
\end{myverb}

Теперь мы можем определить \lstinline{define-binary-class} и \lstinline{define-tagged-binary-class} так, чтобы они раскрывались в вызов \lstinline{define-generic-binary-class}. Вот новая версия \lstinline{define-binary-class}, которая генерирует при полном своём раскрытии тот же код, что и более ранняя:

\begin{myverb}
(defmacro define-binary-class (name (&rest superclasses) slots)
  (with-gensyms (objectvar streamvar)
    `(define-generic-binary-class ,name ,superclasses ,slots
       (defmethod read-object progn ((,objectvar ,name) ,streamvar)
         (declare (ignorable ,streamvar))
         (with-slots ,(new-class-all-slots slots superclasses) ,objectvar
           ,@(mapcar #'(lambda (x) (slot->read-value x streamvar)) slots))))))
\end{myverb}

А вот \lstinline{define-tagged-binary-class} вместе с двумя вспомогательными функциями, которые
он использует:

\begin{myverb}
(defmacro define-tagged-binary-class (name (&rest superclasses) slots &rest options)
  (with-gensyms (typevar objectvar streamvar)
    `(define-generic-binary-class ,name ,superclasses ,slots
      (defmethod read-value ((,typevar (eql ',name)) ,streamvar &key)
        (let* ,(mapcar #'(lambda (x) (slot->binding x streamvar)) slots)
          (let ((,objectvar
                 (make-instance 
                  ,@(or (cdr (assoc :dispatch options))
                        (error "Must supply :dispatch form."))
                  ,@(mapcan #'slot->keyword-arg slots))))
            (read-object ,objectvar ,streamvar)
            ,objectvar))))))

(defun slot->binding (spec stream)
  (destructuring-bind (name (type &rest args)) (normalize-slot-spec spec)
    `(,name (read-value ',type ,stream ,@args))))

(defun slot->keyword-arg (spec)
  (let ((name (first spec)))
    `(,(as-keyword name) ,name)))
\end{myverb}

\section{Примитивные двоичные типы}

Хотя \lstinline{define-binary-class} и \lstinline{define-tagged-binary-class} делают лёгким
определение сложных структур, мы все ещё должны написать методы \lstinline{read-value} и
\lstinline{write-value} для примитивных типов данных вручную. Мы могли бы смириться с этим,
специфицировав, что пользователь библиотеки должен написать соответствующие методы
\lstinline{read-value} и \lstinline{write-value} для поддержки примитивных типов, используемых его
двоичными классами.

Однако вместо документирования того, как написать подходящую пару методов
\lstinline{read-value}/\lstinline{write-value}, мы можем предоставить макрос, делающий это
автоматически. Это также даст пользу уменьшения <<протечки>> абстракции, созданной
\lstinline{define-binary-class}. Сейчас \lstinline{define-binary-class} зависит от наличия методов
\lstinline{read-value} и \lstinline{write-value}, определённых неким специфическим образом, который
в действительности является деталью реализации. Определив макрос генерации методов
\lstinline{read-value} и \lstinline{write-value} для примитивных типов, мы скроем эти детали за
управляемой нами абстракцией. Приняв позднее решение изменить реализацию
\lstinline{define-binary-class}, мы сможем изменить наш макрос определения примитивных типов
так, что не потребуется вносить изменения в код, использующий нашу библиотеку двоичных
данных.

Итак, мы должны определить последний макрос, \lstinline{define-binary-type}, который будет
генерировать методы \lstinline{read-value} и \lstinline{write-value} для чтения и записи значений,
предствляющих экземпляры уже существующих, а не определённых с помощью
\lstinline{define-binary-class}, классов.

В качестве конкретного примера рассмотрим тип, используемый классом \lstinline{id3-tag}: строку
знаков фиксированной длины, закодирированную с помощью ISO-8859-1. Я подразумеваю, как
делал это и раньше, что внутренней кодировкой вашей реализации Lisp является ISO-8859-1
или её надмножество, так что вы можете использовать функции \lstinline{CODE-CHAR} и
\lstinline{CHAR-CODE} для преобразования байтов в знаки и обратно.

Как всегда, нашей целью является написание макроса, который позволит нам ограничиться
выражением только самой существенной информации, необходимой для генерации требуемого
кода. В данном случае есть четыре части такой существенной информации: имя типа,
\lstinline{iso-8859-1-string}; \lstinline!&key! параметры, которые должны приниматься методами
\lstinline{read-value} и \lstinline{write-value}, в нашем случае это \lstinline{length}; код считывания
из потока; код записи в поток. Вот выражение, содержащее все четыре части информации:

\begin{myverb}
(define-binary-type iso-8859-1-string (length)
  (:reader (in)
    (let ((string (make-string length)))
      (dotimes (i length)
        (setf (char string i) (code-char (read-byte in))))
      string))
  (:writer (out string)
    (dotimes (i length)
      (write-byte (char-code (char string i)) out))))
\end{myverb}

Теперь все, что нам нужно,~--- так это макрос, осуществляющий разбор такой формы и
преобразующий её в две \lstinline{DEFMETHOD}, обернутые в форму \lstinline{PROGN}. Если мы
определим список параметров \lstinline{define-binary-type} следующим образом:

\begin{myverb}
(defmacro define-binary-type (name (&rest args) &body spec) ...
\end{myverb}

\noindent{}то внутри макроса параметр \lstinline{spec} будет списком, содержащим определения процедур
чтения и записи. Вы можете использовать функцию \lstinline{ASSOC} для извлечения элементов
\lstinline{spec} с помощью меток \lstinline{:reader} и \lstinline{:writer}, а затем
\lstinline{DESTRUCTURING-BIND} для разбора \lstinline{REST}-части каждого
элемента\footnote{Использование \lstinline{ASSOC} для извлечения элементов \lstinline{:reader} и
  \lstinline{:writer} параметра \lstinline{spec} позволяет пользователям макроса
  \lstinline{define-binary-type} включать эти элементы в любом порядке; решив, что элемент
  \lstinline{:reader} всегда будет на первом месте, мы могли бы использовать 
\lstinline{(rest (first spec))} для извлечения части для чтения данных и \lstinline{(rest (second spec))} для
извлечения части для их записи. Однако, так как мы решили использовать ключевые слова
\lstinline{:reader} и \lstinline{:writer} для улучшения читаемости форм \lstinline{define-binary-type},
мы также можем использовать их для извлечения правильных данных.}.

Остальная работа является всего лишь делом подстановки извлечённых значений в шаблоны
квазицитирования (backquoted templates) методов \lstinline{read-value} и \lstinline{write-value}.

\begin{myverb}
(defmacro define-binary-type (name (&rest args) &body spec)
  (with-gensyms (type)
    `(progn
      ,(destructuring-bind ((in) &body body) (rest (assoc :reader spec))
        `(defmethod read-value ((,type (eql ',name)) ,in &key ,@args)
          ,@body))
      ,(destructuring-bind ((out value) &body body) (rest (assoc :writer spec))
        `(defmethod write-value ((,type (eql ',name)) ,out ,value &key ,@args)
          ,@body)))))
\end{myverb}

Обратите внимание на вложенность шаблонов квазицитирования: самый внешний шаблон
начинается с закавыченной формы \lstinline{PROGN}. Этот шаблон состоит из символа
\lstinline{PROGN} и двух <<раскавыченных>> (comma-unquoted) выражений
\lstinline{DESTRUCTURING-BIND}. Таким образом, внешний шаблон заполняется путём вычисления
выражений \lstinline{DESTRUCTURING-BIND} и подстановки значений их результатов. Каждое
выражение \lstinline{DESTRUCTURING-BIND}, в свою очередь, также содержит шаблон
квазицитирования, каждый из которых используется для генерации определения метода для
подстановки его во внешний шаблон.

Теперь данная ранее форма \lstinline{define-binary-type} раскрывается в такой код:

\begin{myverb}
(progn
  (defmethod read-value ((#:g1618 (eql 'iso-8859-1-string)) in &key length)
    (let ((string (make-string length)))
      (dotimes (i length)
        (setf (char string i) (code-char (read-byte in))))
      string))
  (defmethod write-value ((#:g1618 (eql 'iso-8859-1-string)) out string &key length)
    (dotimes (i length)
      (write-byte (char-code (char string i)) out))))
\end{myverb}

Конечно, сейчас, когда у нас есть такой хороший макрос для определения двоичных типов,
привлекательной кажется мысль об улучшении этого макроса таким образом, чтобы он
проделывал больше работы. И нам следует сделать лишь одно небольшое улучшение,
которое окажется весьма полезным, когда мы начнём использовать нашу библиотеку при работе
с реальными форматами файлов, такими как, например, теги ID3.

Теги ID3, подобно многим другим двоичным форматам, используют множество примитивных типов,
являющихся небольшими вариациями на одну тему, такими как беззнаковые целые размерами
один, два, три и четыре байта. Конечно же мы можем определить каждый такой тип с помощью
\lstinline{define-binary-type}. Но мы также можем вынести общий алгоритм чтения и записи
n-байтных беззнаковых целых во вспомогательную функцию.

Но представим, что мы уже определили двоичный тип, \lstinline{unsigned-integer}, который
принимает параметр \lstinline{:bytes} для указания того, как много байт нужно считывать и
записывать. Используя этот тип, мы можем определить слот, представляющий однобайтное целое,
с помощью спецификатора \lstinline{(unsigned-integer :bytes 1)}. Но, если определённый двоичный
формат определяет множество слотов этого типа, было бы неплохо иметь возможность лёгкого
определения нового типа, скажем \lstinline{u1}, означающего то же, что и используемый тип. На
самом деле несложно изменить \lstinline{define-binary-type} так, чтобы он поддерживал две
формы: длинную форму, состоящую из пары \lstinline{:reader} и \lstinline{:writer}, и короткую форму,
которая определяет новый двоичный тип в терминах уже су\-щест\-вую\-щего типа. Используя
короткую форму \lstinline{define-binary-type}, вы можете определить \lstinline{u1} следующим образом:

\begin{myverb}
(define-binary-type u1 () (unsigned-integer :bytes 1))
\end{myverb}

\noindent{}что раскроется в такое:

\begin{myverb}
(progn
  (defmethod read-value ((#:g161887 (eql 'u1)) #:g161888 &key)
    (read-value 'unsigned-integer #:g161888 :bytes 1))
  (defmethod write-value ((#:g161887 (eql 'u1)) #:g161888 #:g161889 &key)
    (write-value 'unsigned-integer #:g161888 #:g161889 :bytes 1)))
\end{myverb}

Для поддержки и длинной, и короткой форм \lstinline{define-binary-type} нам нужно различать эти
два случая, основываясь на значении аргумента \lstinline{spec}. Если \lstinline{spec} содержит два
элемента, он представляет длинную форму, а эти два элемента должны быть определениями
\lstinline{:reader} и \lstinline{:writer}, извлечение которых было реализовано нами раньше. Если же
этот аргумент содержит лишь один элемент, этот элемент должен быть спецификатором типа,
разбор которого будет отличаться. Мы можем использовать \lstinline{ECASE} для осуществления
дифференциации по длине аргумента \lstinline{spec} с целью дальнейшего осуществления разбора
этого аргумента и генерации соответствующего форме (длинной или короткой) раскрытия.

\begin{myverb}
(defmacro define-binary-type (name (&rest args) &body spec)
  (ecase (length spec)
    (1
     (with-gensyms (type stream value)
       (destructuring-bind (derived-from &rest derived-args) (mklist (first spec))
         `(progn
            (defmethod read-value ((,type (eql ',name)) ,stream &key ,@args)
              (read-value ',derived-from ,stream ,@derived-args))
            (defmethod write-value ((,type (eql ',name)) ,stream ,value &key ,@args)
              (write-value ',derived-from ,stream ,value ,@derived-args))))))
    (2
     (with-gensyms (type)
       `(progn
          ,(destructuring-bind ((in) &body body) (rest (assoc :reader spec))
             `(defmethod read-value ((,type (eql ',name)) ,in &key ,@args)
                ,@body))
          ,(destructuring-bind ((out value) &body body) (rest (assoc :writer spec))
             `(defmethod write-value ((,type (eql ',name)) ,out ,value &key ,@args)
                ,@body)))))))
\end{myverb}

\section{Стек обрабатываемых в данный момент объектов}

Последней частью функциональности, которая нам понадобится в следующей главе, является
возможность обращения к двоичному объекту, чтение или запись которого производится в
данный момент. То есть во время чтения или записи вложенных сложных объектов было бы
удобно иметь возможность получения доступа к объекту, чтение или запись которого
производится в данный момент. Благодаря динамическим переменным и методам \lstinline{:around}
мы можем добавить это улучшение, написав всего лишь около дюжины строк кода. Для начала
нам нужно определить динамическую переменную, которая будет содержать стек объектов,
чтение или запись которых осуществляется в данный момент.

\begin{myverb}
(defvar *in-progress-objects* nil)
\end{myverb}

Затем мы можем определить методы \lstinline{:around} для \lstinline{read-object} и
\lstinline{write-object}, которые помещают объект, чтение или запись которого будет
осуществляться, в определённую ранее переменную перед вызовом \lstinline{CALL-NEXT-METHOD}.

\begin{myverb}
(defmethod read-object :around (object stream)
  (declare (ignore stream))
  (let ((*in-progress-objects* (cons object *in-progress-objects*)))
    (call-next-method)))

(defmethod write-object :around (object stream)
  (declare (ignore stream))
  (let ((*in-progress-objects* (cons object *in-progress-objects*)))
    (call-next-method)))
\end{myverb}

Обратите внимание, как мы пересвязали \lstinline{*in-progress-objects*} со списком, содержащим
новый элемент в своём начале, вместо присвоения ему нового значения. Поступив так, мы
получили тот эффект, что в конце \lstinline{LET}, после возврата из
\lstinline{CALL-NEXT-METHOD}, старое значение \lstinline{*in-progress-objects*} будет
восстановлено (то есть последний помещённый в стек элемент будет из него удалён).

Имея эти определения методов, мы можем предоставить две удобные функции для получения
отдельных объектов из стека обрабатываемых объектов. Функция \lstinline{current-binary-object}
будет возвращать вершину стека, то есть объект, чей метод \lstinline{read-object} или
\lstinline{write-object} был вызван наиболее недавно. Вторая, \lstinline{parent-of-type}, получает
аргумент, который должен быть именем класса двоичного типа, и возвращает наиболее недавно
помещённый в стек объект данного типа, используя функцию \lstinline{TYPEP}, которая
проверяет, является ли переданный ей объект экземпляром определённого типа.

\begin{myverb}
(defun current-binary-object () (first *in-progress-objects*))

(defun parent-of-type (type)
  (find-if #'(lambda (x) (typep x type)) *in-progress-objects*))
\end{myverb}

Эти две функции могут быть использованы из любого кода, который будет вызван в
динамической протяжённости вызова \lstinline{read-object} или \lstinline{write-object}. Мы увидим
один пример использования \lstinline{current-binary-object} в следующей главе\footnote{Формат
  ID3 не требует использования функции \lstinline{parent-of-type}, так как имеет сравнительно
  <<плоскую>> структуру. Эта функция становится очень полезной, если вы осуществляете
  разбор формата, состоящего из множества глубоко вложенных структур, чей разбор зависит от
  информации, сохранённой в структурах более высокого уровня. Например, в формате файлов
  классов Java структура файлов классов верхнего уровня содержит пул констант,
  отображающий числовые значения, используемые в других подструктурах внутри файла класса,
  на константные значения, которые нужны во время разбора этих подструктур. Если бы мы писали
  программу разбора файлов классов, то могли бы использовать \lstinline{parent-of-type} в коде
  чтения и записи этих подструктур для обращения к объекту файла класса верхнего уровня, а
  по нему~--- к пулу констант.}.

Теперь у нас есть все инструменты, нужные для создания библиотеки разбора ID3, поэтому мы
готовы перейти к следующей главе, где мы именно этим и займёмся.

%%% Local Variables: 
%%% mode: latex
%%% TeX-master: "pcl-ru"
%%% TeX-open-quote: "<<"
%%% TeX-close-quote: ">>"
%%% End: 
