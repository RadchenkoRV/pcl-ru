\chapter{Коллекции}
\label{ch:11}

Как и в большинстве языков программирования, в Common Lisp есть стандартные типы данных,
собирающие несколько значений в один объект.  Каждый язык решает проблему коллекций
немного по-разному, но базовые типы коллекций обычно сводятся к типу массивов с целочисленными
индексами и типу таблиц, способных отображать более или менее
произвольные ключи в значения.  Первые называются массивами, списками или кортежами, а
вторые~--- хэш-таблицами, ассоциативными массивами, картами и словарями.

Конечно, Lisp знаменит своими списками, поэтому, согласно принципу <<Онтогенез повторяет
филогенез>>, большинство учебников по Lisp начинают объяснение коллекций со
списков. Однако такой подход часто приводит читателей к ошибочному выводу, что
список~--- единственный тип коллекций в Lisp.  Что ещё хуже,
списки Lisp~--- структуры настолько гибкие, что их можно использовать для многих целей, для
которых в других языках используются массивы и хэш-таблицы.  Но было бы ошибкой слишком
сильно сосредотачиваться на списках; хотя они и являются ключевой структурой данных для
представления Lisp-кода в виде Lisp-данных, во многих случаях более уместны другие структуры данных.

Чтобы списки не затмили всё остальное, в этой главе я сосредоточусь на других типах
коллекций Common Lisp: векторах и хэш-таблицах.\footnote{Когда вы познакомитесь со всеми
  типами данных в Common Lisp, вы увидите, что может быть полезно сначала моделировать
  структуру данных с помощью списков, а затем заменять списки на что-то более 
  эффективное, после того, как станет ясно, как именно данные будут использоваться.}
Однако векторы и списки имеют достаточно много общих признаков, так что Common Lisp
рассматривает их как подтипы более общей абстракции~--- последовательности.  Таким образом,
многие функции, описанные в этой главе, можно использовать как для векторов, так и для
списков.

\section{Векторы}

Векторы~--- основной тип коллекций с доступом по целочисленному индексу в Common Lisp, и они
имеют две разновидности.  Векторы с фиксированным размером похожи на массивы в языках,
подобных Java: простая надстройка над непрерывной областью памяти, хранящей элементы
вектора.\footnote{Векторы называются векторами, а не массивами, в отличие от их аналогов в других
  языках программирования, поскольку Common Lisp поддерживает настоящие многомерные
  массивы.  Одинаково корректно (хотя и несколько неуклюже) ссылаться на них как на
  одномерные массивы} С другой стороны, векторы с изменяемым размером более похожи на
векторы в Perl или Ruby, списки в Python, или на класс \code{ArrayList} в Java: они скрывают
конкретный способ хранения, позволяя векторам менять размер по мере добавления или
удаления элементов.

Вы можете создать вектор фиксированной длины, содержащий конкретные значения, с помощью
функции \code{VECTOR}, которая принимает любое количество аргументов и возвращает
новый вектор фиксированного размера, содержащий переданные значения.

\begin{verbatim}
  (vector)     ==> #()
  (vector 1)   ==> #(1)
  (vector 1 2) ==> #(1 2)
\end{verbatim}

Синтаксис \lstinline!#(...)!~--- способ записи векторов, используемый процедурами
записи и чтения Lisp.  Этот синтаксис позволяет вам сохранять и загружать векторы,
выводя их на печать с помощью \code{PRINT} и считывая с помощью code{READ}.
Вы можете использовать синтаксис \lstinline!#(...)! для
записи векторных литералов в вашем коде, но поскольку эффект изменения литералов не определён,
то для создания векторов, которые планируется изменять, вы всегда должны 
использовать \code{VECTOR} или более общую функцию \code{MAKE-ARRAY}.

\code{MAKE-ARRAY}~--- более общая функция, чем \code{VECTOR}, поскольку её можно
использовать для создания массивов любой размерности, а также для создания векторов
фиксированной и изменяемой длины. Единственным обязательным аргументом \code{MAKE-ARRAY}
является список, содержащий размерности массива.  Поскольку вектор~--- одномерный массив,
то список будет содержать только одно число~--- размер вектора.  Для удобства,
\code{MAKE-ARRAY} может также принимать обычное число вместо списка из одного элемента.
Без предоставления дополнительных аргументов, \code{MAKE-ARRAY} создаст вектор с
неинициализированными элементами, которые перед использованием будет необходимо
проинициализировать.\footnote{Элементы массива <<должны>> быть заданы до того, как вы будете осуществлять
  доступ к ним, поскольку иначе поведение будет неопределённым; Lisp не обязан
  останавливать вас при совершении ошибок.}  Для создания вектора, с присвоением всем
элементам определённого значения, вы можете использовать аргумент \code{:initial-element}.
Таким образом, создать вектор из пяти элементов, равных \code{NIL}, можно так:

\begin{verbatim}
  (make-array 5 :initial-element nil) ==> #(NIL NIL NIL NIL NIL)
\end{verbatim}

\code{MAKE-ARRAY} также используется для создания векторов переменного размера.
Вектор с изменяемым размером несколько сложнее, чем вектор фиксированного размера;
помимо количества доступных ячеек и области памяти, выделенной под хранение элементов,
вектор с изменяемым размером также отслеживает, сколько элементов фактически
хранится в векторе.  Это число хранится в указателе заполнения вектора (vector's fill
pointer), названного так, поскольку это индекс позиции, которая будет заполнена следующей,
когда вы добавите элемент в вектор.

Чтобы создать вектор с указателем заполнения, вы должны передать \code{MAKE-ARRAY}
аргумент \code{:fill-pointer}.  Например, следующий вызов \code{MAKE-ARRAY} создаст вектор
с местом для пяти элементов; но он будет выглядеть пустым, поскольку указатель заполнения
равен нулю:

\begin{verbatim}
  (make-array 5 :fill-pointer 0) ==> #()
\end{verbatim}

Чтобы добавить элемент в конец вектора, можно использовать функцию
\code{VECTOR-PUSH}.  Она добавляет элемент в позицию, указываемую указателем заполнения, и
затем увеличивает его на единицу, возвращая индекс ячейки, куда был добавлен новый
элемент.  Функция \code{VECTOR-POP} возвращает последний добавленный элемент, уменьшая
указатель заполнения на единицу.

\begin{verbatim}
  (defparameter *x* (make-array 5 :fill-pointer 0))
  
  (vector-push 'a *x*) ==> 0
  *x*                  ==> #(A)
  (vector-push 'b *x*) ==> 1
  *x*                  ==> #(A B)
  (vector-push 'c *x*) ==> 2
  *x*                  ==> #(A B C)
  (vector-pop *x*)     ==> C
  *x*                  ==> #(A B)
  (vector-pop *x*)     ==> B
  *x*                  ==> #(A)
  (vector-pop *x*)     ==> A
  *x*                  ==> #()
\end{verbatim}

Однако даже вектор с указателем заполнения на самом деле не является вектором с изменяемыми
размерами.  Вектор \code{*x*} может хранить максимум пять элементов.  Для того, чтобы
создать вектор с изменяемым размером, вам необходимо передать \code{MAKE-ARRAY} другой
именованный аргумент: \code{:adjustable}.

\begin{verbatim}
  (make-array 5 :fill-pointer 0 :adjustable t) ==> #()
\end{verbatim}

Этот вызов создаст вектор, размер которого может изменяться по мере необходимости.  Чтобы
добавить элементы в такой вектор, вам нужно использовать функцию
\code{VECTOR-PUSH-EXTEND}, которая работает так же, как и \code{VECTOR-PUSH}, однако
она автоматически увеличит массив, если вы попытаетесь добавить элемент в
уже заполненный вектор~--- вектор, чей указатель заполнения равен размеру выделенной
памяти.\footnote{Хотя аргументы \code{:fill-pointer} и \code{:adjustable} часто используются
  вместе, но они независимы друг от друга~--- вы можете создать
  массив с изменяемым размером без указателя заполнения.  Однако вы можете использовать
  \code{VECTOR-PUSH} и \code{VECTOR-POP} только с векторами, которые имеют указатель
  заполнения, а \code{VECTOR-PUSH-EXTEND}~--- только с векторами, которые имеют переменный
  размер и указатель заполнения.  Вы также можете использовать функцию \code{ADJUST-ARRAY}
  для изменения параметров массивов переменной длины, а не только изменения длины
  вектора.}


\section{Подтипы векторов}

Все векторы, с которыми мы уже встречались, были векторами общего назначения, которые
могут хранить объекты любого типа.  Однако возможно и создание специализированных
векторов, которые предназначены для хранения элементов определённого типа.  Одной из
причин использовать специализированные векторы является то, что они могут требовать
меньше памяти и обеспечивать более быстрый доступ к своим элементам, по сравнению с
векторами общего назначения.  Однако давайте сосредоточимся на некоторых
специализированных векторах, которые сами по себе являются важными типами данных.

С одним из них мы уже встречались: строки~--- это векторы, предназначенные для хранения
знаков.  Строки так важны, что для них предусмотрен собственный синтаксис чтения/записи
(двойные кавычки) и набор отдельных функций, которые мы обсуждали в предыдущей главе.  Но,
поскольку они также являются векторами, то все функции, работающие с векторами и которые мы
будем обсуждать в следующих разделах, могут также использоваться для работы со строками.
Эти функции дополнят библиотеку функций работы со строками новыми функциями для таких
операций, как поиск подстроки в строке, нахождение позиции знака в строке и т.п.

Строки, такие как \code{"foo"}, подобны векторам, записанным с использованием синтаксиса
\lstinline!#()!~--- их размер фиксирован, и они не должны изменяться.  Однако вы можете
использовать функцию \code{MAKE-ARRAY} для создания строк с изменяемым размером, просто
добавив ещё один именованный аргумент~--- \code{:element-type}.  Этот аргумент принимает
описание типа.  Я не буду тут описывать типы, которые вы можете использовать; сейчас
достаточно знать, что вы можете создать строку, передав символ \code{CHARACTER} в
качестве аргумента \code{:element-type}.  Заметьте, что необходимо экранировать
символ, чтобы он не считался именем переменной.  Например, чтобы создать пустую строку с
изменяемым размером, вы можете написать вот так:

\begin{verbatim}
  (make-array 5 :fill-pointer 0 :adjustable t :element-type 'character)  ""
\end{verbatim}

Битовые векторы (специализированные векторы, чьи элементы могут иметь значение ноль или
один) также отличаются от обычных векторов.  Они также имеют специальный синтаксис
чтения/записи, который выглядит вот так \lstinline!#*00001111!, а также, достаточно большой
набор функций (которые я не буду тут описывать) для выполнения битовых операций, таких как
выполнение <<и>> для двух битовых массивов.  Для создания такого вектора нужно передать
в качестве \code{:element-type} символ \code{BIT}.

\section{Векторы как последовательности}

Как уже упоминалось ранее, векторы и списки являются подтипами абстрактного типа
<<последовательность>>.  Все функции, которые будут обсуждаться в следующих разделах,
работают с последовательностями; они могут работать не только с векторами (и
специализированным, и общего назначения), но и со списками.

Две самых простых функции для работы с последовательностями~--- \code{LENGTH},
возвращающая длину последовательности, и \code{ELT}, осуществляющая доступ к
отдельным элементам по целочисленному индексу.  \code{LENGTH} получает
последовательность в качестве единственного аргумента и возвращает число элементов в этой
последовательности.  Для векторов с указателем заполнения это число будет равно значению
указателя. \code{ELT} (сокращение слова элемент) получает два аргумента~---
последовательность и числовой индекс между нулём (включительно) и длиной
последовательности (не включительно) и возвращает соответствующий элемент.  \code{ELT} выдаст ошибку,
если индекс находится за границами последовательности.  Подобно \code{LENGTH}, \code{ELT}
рассматривает вектор с указателем заполнения, как имеющий длину, заданную этим
указателем.

\begin{verbatim}
  (defparameter *x* (vector 1 2 3))
  
  (length *x*) ==> 3
  (elt *x* 0)  ==> 1
  (elt *x* 1)  ==> 2
  (elt *x* 2)  ==> 3
  (elt *x* 3)  ==> error
\end{verbatim}

\code{ELT} возвращает ячейку, для которой можно выполнить \code{SETF}, так что вы можете
установить значение отдельного элемента с помощью вот такого кода:

\begin{verbatim}
  (setf (elt *x* 0) 10)
  
  *x* ==> #(10 2 3)
\end{verbatim}

\section{Функции для работы с элементами последовательностей}

Хотя в теории все операции над последовательностями могут быть сведены к комбинациям
\code{LENGTH}, \code{ELT}, и \code{SETF} на результат \code{ELT}, Common Lisp все равно
предоставляет большую библиотеку функций для работы с последовательностями.

Одна группа функций позволит вам выполнить некоторые операции, такие как нахождение или
удаление определённых элементов, без явного написания циклов.  Краткая сводка этих функций
приводится в таблице~\ref{table:11-1}.

\begin{figure}[tb]
\begin{tabular}{|>{\centering}m{25mm}|>{\centering}m{25mm}|>{\centering}m{25mm}|}
Название &Обязательные аргументы &Возвращаемое значение \\
\code{COUNT}       &Объект и последовательность  &Число вхождений в последовательности\\
\code{FIND}        &Объект и последовательность  &Объект или \code{NIL}\\
\code{POSITION}    &Объект и последовательность  &Индекс ячейки в последовательности или \code{NIL}\\
\code{REMOVE}      &Удаляемый объект и последовательность  &Последовательность, из которой удалены указанные объекты\\
\code{SUBSTITUTE}  &Новый объект, заменяемый объект и последовательность &Последовательность, в которой указанные объекты заменены на новые
\end{tabular}
  \caption{Базовые функции для работы с последовательностями} 
  \label{table:11-1}
\end{figure}

Вот несколько простых примеров использования этих функций:

\begin{verbatim}
  (count 1 #(1 2 1 2 3 1 2 3 4))         ==> 3
  (remove 1 #(1 2 1 2 3 1 2 3 4))        ==> #(2 2 3 2 3 4)
  (remove 1 '(1 2 1 2 3 1 2 3 4))        ==> (2 2 3 2 3 4)
  (remove #\a "foobarbaz")               ==> "foobrbz"
  (substitute 10 1 #(1 2 1 2 3 1 2 3 4)) ==> #(10 2 10 2 3 10 2 3 4)
  (substitute 10 1 '(1 2 1 2 3 1 2 3 4)) ==> (10 2 10 2 3 10 2 3 4)
  (substitute #\x #\b "foobarbaz")       ==> "fooxarxaz"
  (find 1 #(1 2 1 2 3 1 2 3 4))          ==> 1
  (find 10 #(1 2 1 2 3 1 2 3 4))         ==> NIL
  (position 1 #(1 2 1 2 3 1 2 3 4))      ==> 0
\end{verbatim}

Заметьте, что \code{REMOVE} и \code{SUBSTITUTE} всегда возвращают последовательность
того-же типа, что и переданный аргумент.

Вы можете изменить поведение этих функций с помощью различных именованных аргументов.
Например, по умолчанию эти функции ищут в последовательности точно такой же объект, что и
переданный в качестве аргумента.  Вы можете изменить это поведение двумя способами: во
первых, вы можете использовать именованный аргумент \code{:test} для указания функции,
которая принимает два аргумента, и возвращает логическое значение.  Если этот аргумент
указан, то он будет использоваться для сравнения каждого элемента с эталонным, вместо стандартной
проверки на равенство с помощью \code{EQL}.\footnote{Другой именованный параметр,
  \code{:test-not} указывает предикат, который будет использоваться точно также как и
  параметр \code{:test}, но результат будет изменён на
  противоположное значение.  Однако этот параметр считается устаревшим, и предпочтительным
  является использование функции \code{COMPLEMENT}.  \code{COMPLEMENT} получает
  аргумент-функцию и возвращает функцию, которая получает то же самое количество
  аргументов, что и оригинальная, но возвращает результат, имеющий противоположное
  значение результату возвращаемому оригинальной функцией.  Так что вы можете (и должны)
  писать вот так:

\begin{verbatim}
  (count x sequence :test (complement #'some-test))
\end{verbatim}

вместо:

\begin{verbatim}
  (count x sequence :test-not #'some-test)
\end{verbatim}

} Во-вторых, используя именованный параметр \code{:key}, вы можете передать функцию одного
аргумента, с помощью которой из каждого элемента последовательности будет извлекаться ключ,
который затем будет сравниваться с переданным объектом.  Однако заметьте, что
функции (например \code{FIND}), возвращающие элементы последовательности, все равно будут
возвращать элементы, а не ключи, извлечённые из этих элементов.

\begin{verbatim}
  (count "foo" #("foo" "bar" "baz") :test #'string=)    ==> 1
  (find 'c #((a 10) (b 20) (c 30) (d 40)) :key #'first) ==> (C 30)
\end{verbatim}

Для ограничения действия этих функций в рамках только определённых пределов вы можете
указать граничные индексы, используя именованные аргументы \code{:start} и \code{:end}.
Передача \code{NIL} в качестве значения \code{:end} (или его полное отсутствие)
равносильно указанию длины последовательности.\footnote{Заметьте, однако, что для
  \code{REMOVE} и \code{SUBSTITUTE} указание \code{:start} и \code{:end} приводит к
  ограничению количества аргументов, подпадающих под удаление или замену; элементы до
  \code{:start} и после \code{:end} будут переданы без изменений.}

Если указывается не равный \code{NIL} аргумент \code{:from-end}, то элементы
последовательности проверяются в обратном порядке.  Простое указание \code{:from-end}
может затронуть результаты \code{FIND} и \code{POSITION}.  Например:

\begin{verbatim}
  (find 'a #((a 10) (b 20) (a 30) (b 40)) :key #'first)             ==> (A 10)
  (find 'a #((a 10) (b 20) (a 30) (b 40)) :key #'first :from-end t) ==> (A 30)
\end{verbatim}

Также использование \code{:from-end} может влиять на работу \code{REMOVE} и
\code{SUBSTITUTE} при использовании с другим именованным параметром~--- \code{:count},
который используется для указания количества заменяемых или удаляемых элементов.  Если вы
указываете \code{:count} меньший, чем количество совпадающих элементов, то результат будет
зависеть от того, с какого конца последовательности вы начинаете обработку:

\begin{verbatim}
  (remove #\a "foobarbaz" :count 1)             ==> "foobrbaz"
  (remove #\a "foobarbaz" :count 1 :from-end t) ==> "foobarbz"
\end{verbatim}

И хотя \code{:from-end} не может изменить результат функции \code{COUNT}, его
использование может влиять на порядок элементов, передаваемых функциям, указанным в
параметрах \code{:test} и \code{:key}, которые возможно могут вызвать побочные эффекты.
Например:

\begin{verbatim}
  CL-USER> (defparameter *v* #((a 10) (b 20) (a 30) (b 40)))
  *V*
  CL-USER> (defun verbose-first (x) (format t "Looking at ~s~%" x) (first x))
  VERBOSE-FIRST
  CL-USER> (count 'a *v* :key #'verbose-first)
  Looking at (A 10)
  Looking at (B 20)
  Looking at (A 30)
  Looking at (B 40)
  2
  CL-USER> (count 'a *v* :key #'verbose-first :from-end t)
  Looking at (B 40)
  Looking at (A 30)
  Looking at (B 20)
  Looking at (A 10)
  2
\end{verbatim}

В таблице~\ref{table:11-2} приведены описания всех стандартных аргументов.

\begin{figure}[tb]
\begin{tabular}{|>{\centering}m{25mm}|>{\centering}m{25mm}|>{\centering}m{25mm}|}
Аргумент  &Описание   &Значение по умолчанию\\
\code{:test}  &Функция двух аргументов, используемая для сравнения элементов (или их ключей, извлечённых функцией \code{:key}) с указанным объектом.  &\code{EQL}\\
\code{:key} &Функция одного аргумента, используемая для извлечения ключа из элемента последовательности.  \code{NIL} указывает на использование самого элемента. &\code{NIL}\\
\code{:start}  &Начальный индекс (включительно) обрабатываемой последовательности.  &\code{0}\\
\code{:end}  &Конечный индекс (не включительно) обрабатываемой последовательности.  \code{NIL} указывает на конец последовательности. &\code{NIL}\\
\code{:from-end}  &Если имеет истинное значение, то последовательность будет обрабатываться в обратном порядке, от конца к началу. &\code{NIL}\\
\code{:count} &Число, указывающее количество удаляемых или заменяемых элементов, или \code{NIL} для всех элементов (только для \code{REMOVE} и \code{SUBSTITUTE}). &\code{NIL}
\end{tabular}
  \caption{Стандартные именованные аргументы функций работы с последовательностями} 
  \label{table:11-2}
\end{figure}

\section{Аналогичные функции высшего порядка}

Для каждой из функций, которая была описана выше, Common Lisp также предоставляет два
набора функций высшего порядка, которые вместо аргумента, используемого для сравнения,
получают функцию, которая вызывается для каждого элемента последовательности.  Первый
набор функций имеет те же имена, что и функции из базового набора, но с добавлением
суффикса \code{-IF}.  Эти функции подсчитывают, ищут, удаляют и заменяют элементы, для
которых аргумент-функция возвращает истинное значение.  Другой набор функций, отличается
тем, что использует суффикс \code{-IF-NOT}, и выполняет те же операции, но для элементов,
для которых функция не возвращает истинного значения.

\begin{verbatim}
  (count-if #'evenp #(1 2 3 4 5))         ==> 2

  (count-if-not #'evenp #(1 2 3 4 5))     ==> 3

  (position-if #'digit-char-p "abcd0001") ==> 4

  (remove-if-not #'(lambda (x) (char= (elt x 0) #\f))
    #("foo" "bar" "baz" "foom")) ==> #("foo" "foom")
\end{verbatim}

В соответствии со стандартом языка, функции с суффиксом \code{-IF-NOT} являются
устаревшими.  Однако это требование само считается неразумным.  Если стандарт будет
пересматриваться, то скорее будет удалено это требование, а не функции с суффиксом
\code{-IF-NOT}.  Для некоторых вещей, \code{REMOVE-IF-NOT} может использоваться чаще, чем
\code{REMOVE-IF}.  За исключением своего отрицательно звучащего имени, в действительности
\code{REMOVE-IF-NOT} является положительной функцией~--- она возвращает элементы, которые
соответствуют предикату.\footnote{Эта функция обеспечивает туже функциональность, что и
  \code{grep} в Perl и \code{filter} в Python.}

Оба варианта функций принимают те же именованные аргументы, что и базовые функции, за
исключением аргумента \code{:test}, который не нужен, поскольку главный аргумент сам
является функцией.\footnote{Отличием предиката, передаваемого аргументу \code{:test} от
  аргумента-функции, передаваемого в функции с суффиксами \code{-IF} и \code{-IF-NOT},
  является то, что предикат параметра \code{:test} имеет два аргумента и используется для
  сравнения элементов последовательности с конкретным объектом, в то время как предикаты
  для функций с суффиксами \code{-IF} и \code{-IF-NOT} имеют один аргумент, и используются
  для проверки только элементов последовательности.  Если бы базовые варианты не
  существовали, то вы могли бы реализовать их с помощью функций с суффиксом \code{-IF},
  путём указания объекта в функции-предикате.

\begin{verbatim}
  (count char string) ===
    (count-if #'(lambda (c) (eql char c)) string)
  
  (count char string :test #'CHAR-EQUAL) ===
    (count-if #'(lambda (c) (char-equal char c)) string)
\end{verbatim}

}  При указании аргумента \code{:key}, функции передаётся значение, извлечённое функцией
аргумента \code{:key}, а не сам элемент.

\begin{verbatim}
  (count-if #'evenp #((1 a) (2 b) (3 c) (4 d) (5 e)) :key #'first)     ==> 2

  (count-if-not #'evenp #((1 a) (2 b) (3 c) (4 d) (5 e)) :key #'first) ==> 3

  (remove-if-not #'alpha-char-p
    #("foo" "bar" "1baz") :key #'(lambda (x) (elt x 0))) ==> #("foo" "bar")
\end{verbatim}

Семейство функций \code{REMOVE} также имеет четвёртый вариант, функцию
\code{REMOVE-DUPLICATES}, которая имеет один аргумент~--- последовательность, из которой
удаляются все, кроме одного экземпляра, каждого дублированного элемента.  Она может
принимать те же самые именованные аргументы что и \code{REMOVE}, за исключением
\code{:count}, поскольку она всегда удаляет все дубликаты.

\begin{verbatim}
  (remove-duplicates #(1 2 1 2 3 1 2 3 4)) ==> #(1 2 3 4)
\end{verbatim}

\section{Работа с последовательностью целиком}

Несколько функций оперируют над последовательностями целиком.  Обычно
они проще чем функции, которые я описывал ранее.  Например,
\code{COPY-SEQ} и \code{REVERSE} получают по одному аргументу~--- последовательности, и
возвращают новую последовательность того же самого типа.  Последовательность, возвращённая
\code{COPY-SEQ}, содержит те же самые элементы, что и исходная,
а последовательность, возвращённая \code{REVERSE}, содержит те же
самые элементы, но в обратном порядке.  Заметьте, что ни одна из этих функций не копирует
сами элементы, заново создаётся только возвращаемая последовательность.

Функция \code{CONCATENATE} создаёт новую последовательность, содержащую объединение произвольного
числа последовательностей.  Однако, в отличие от \code{REVERSE} и \code{COPY-SEQ}, которые
просто возвращают последовательность того же типа, что и переданный аргумент, функции
\code{CONCATENATE} должно быть явно указано, какой тип последовательности необходимо
создать в том случае, если её аргументы имеют разные типы.  Первым аргументом функции
является описание типа, подобный параметру \code{:element-type} функции \code{MAKE-ARRAY}.
В этом случае, вероятнее всего, вы будет использовать следующие символы для указания типа:
\code{VECTOR}, \code{LIST} или \code{STRING}.\footnote{Если вы указываете функции
  \code{CONCATENATE}, что она должна вернуть специализированный вектор, например, строку,
  то все элементы аргументов-последовательностей должны иметь тот же тип, что и элементы
  этого вектора.}  Например:

\begin{verbatim}
  (concatenate 'vector #(1 2 3) '(4 5 6))    ==> #(1 2 3 4 5 6)
  (concatenate 'list #(1 2 3) '(4 5 6))      ==> (1 2 3 4 5 6)
  (concatenate 'string "abc" '(#\d #\e #\f)) ==> "abcdef"
\end{verbatim}

\section{Сортировка и слияние}

Функции \code{SORT} и \code{STABLE-SORT} обеспечивают два метода сортировки
последовательности.  Они обе получают последовательность и функцию двух аргументов, и
возвращают отсортированную последовательность.

\begin{verbatim}
  (sort (vector "foo" "bar" "baz") #'string<) ==> #("bar" "baz" "foo")
\end{verbatim}

Разница между этими функциями заключается в том, что \code{STABLE-SORT} гарантирует, что
она не будет изменять порядок элементов, которые считаются эквивалентными, в то время как
\code{SORT} гарантирует только, что результат будет отсортирован, так что некоторые
эквивалентные элементы могут быть поменяны местами.

Обе эти функции представляют собой примеры так называемых деструктивных функций.  Деструктивным
функциям разрешено (обычно в целях эффективности) модифицировать переданные аргументы тем
или иным образом.  Отсюда следует две вещи: во-первых, вы обязательно должны что-то сделать с
возвращаемым значением этих функций (например присвоить его переменной, или передать его
другой функции), и во-вторых, если вы ещё планируете работать с передаваемым в
деструктивную функцию аргументом, вы должны передавать его копию, а не сам объект.
Я расскажу о деструктивных функциях более подробно в следующей главе.

Обычно после того, как последовательность отсортирована, её неотсортированная версия
больше не нужна, так что имеет смысл
позволить \code{SORT} и \code{STABLE-SORT} разрушать последовательность в процессе её
сортировки.  Но это значит, что вы должны не забывать писать так:\footnote{Когда
  передаваемая последовательность является вектором, <<разрушение>> означает изменение
  элементов на месте, так что вы можете обойтись без сохранения возвращаемого значения.
  Однако хорошим стилем будет обязательное использование возвращаемого значения,
  поскольку функции сортировки могут изменять списки произвольным образом.}

\begin{verbatim}
  (setf my-sequence (sort my-sequence #'string<))
\end{verbatim}

вместо:

\begin{verbatim}
  (sort my-sequence #'string<)
\end{verbatim}

Обе эти функции принимают именованный аргумент \code{:key}, который, так же как и аргумент
\code{:key} в других функциях работы с последовательностями, должен быть функцией, и
используется для извлечения значений, которые будут передаваться предикату сортировки
вместо оригинальных элементов.  Извлечённые значения используются только для определения
порядка элементов; возвращённая последовательность будет содержать сами элементы, а не
извлечённые значения.

Функция \code{MERGE} принимает две последовательности и функцию-предикат, и возвращает
последовательность, полученную путём слияния двух последовательностей в соответствии с
предикатом.  Она связана с функциями сортировки: если каждая последовательность
уже была отсортирована с использованием того же самого предиката, то и полученная
последовательность также будет отсортирована.  Так же как и функции сортировки,
\code{MERGE} принимает аргумент \code{:key}.  Подобно \code{CONCATENATE}, и по тем же
причинам, первым аргументом \code{MERGE} должно быть описание типа последовательности,
которая будет получена в результате работы.

\begin{verbatim}
  (merge 'vector #(1 3 5) #(2 4 6) #'<) ==> #(1 2 3 4 5 6)
  (merge 'list #(1 3 5) #(2 4 6) #'<)   ==> (1 2 3 4 5 6)
\end{verbatim}

\section{Работа с частями последовательностей}

Еще один набор функций позволяет работать с частями последовательностей.  Основная из
таких функций~--- \code{SUBSEQ}, которая выделяет часть последовательности
начиная с определённого индекса и заканчивая другим индексом или концом
последовательности.  Например:

\begin{verbatim}
  (subseq "foobarbaz" 3)   ==> "barbaz"
  (subseq "foobarbaz" 3 6) ==> "bar"
\end{verbatim}

Для результата \code{SUBSEQ} также можно выполнить \code{SETF}, но таким образом нельзя увеличить
или уменьшить последовательность; если часть последовательности и новое значение имеют
разные длины, то более короткое из них определяет то, как много знаков будет изменено.

\begin{verbatim}
  (defparameter *x* (copy-seq "foobarbaz"))

  (setf (subseq *x* 3 6) "xxx")  ; subsequence and new value are same length
  *x* ==> "fooxxxbaz"

  (setf (subseq *x* 3 6) "abcd") ; new value too long, extra character ignored.
  *x* ==> "fooabcbaz"

  (setf (subseq *x* 3 6) "xx")   ; new value too short, only two characters changed
  *x* ==> "fooxxcbaz"
\end{verbatim}

Вы можете использовать функцию \code{FILL} для заполнения нескольких значений
последовательности одним и тем же значением.  Обязательные аргументы~---
последовательность и значение, которым нужно заполнить элементы.  По умолчанию
заполняется вся последовательность; вы можете использовать именованные аргументы
\code{:start} и \code{:end} для ограничения границ заполнения.

Если вам нужно найти одну последовательность внутри другой, то вы можете использовать
функцию \code{SEARCH}, которая работает также как и функция \code{POSITION}, но первым
аргументом является последовательность, а не единичное значение.

\begin{verbatim}
  (position #\b "foobarbaz") ==> 3
  (search "bar" "foobarbaz") ==> 3
\end{verbatim}

Для того, чтобы, напротив, найти позицию, в которой две последовательности с общим
префиксом начинают различаться, вы можете использовать функцию \code{MISMATCH}.  Она
принимает две последовательности и возвращает индекс первой пары неравных элементов.

\begin{verbatim}
  (mismatch "foobarbaz" "foom") ==> 3
\end{verbatim}

Эта функция возвращает \code{NIL}, если строки совпадают. \code{MISMATCH} может также
принимать стандартные именованные аргументы: аргумент \code{:key} для указания функции для
извлечения сравниваемых значений; аргумент \code{:test} для указания функции сравнения; и
аргументы \code{:start1}, \code{:end1}, \code{:start2} и :\code{end2} для указания границ
действия внутри последовательностей.  Также, указание \code{:from-end} со значением
\code{T} приводит к тому, что поиск осуществляется в обратном порядке, заставляя
\code{MISMATCH} вернуть индекс позиции в первой последовательности, где начинается общий
суффикс последовательностей.

\begin{verbatim}
  (mismatch "foobar" "bar" :from-end t) ==> 3
\end{verbatim}

\section{Предикаты для последовательностей}

Также существуют полезные функции \code{EVERY}, \code{SOME}, \code{NOTANY} и
\code{NOTEVERY}, которые пробегают по элементам последовательности, проверяя заданный
предикат.  Первым аргументом всех этих функций является предикат, а остальные аргументы~---
последовательности.  Предикат должен получать столько аргументов, сколько
последовательностей будет передано функциям.  Элементы последовательностей передаются
предикату (по одному элементу за раз), пока не закончатся элементы или не будет выполнено
условие завершения: \code{EVERY} завершается, возвращая ложное значение, сразу, как только это
значение будет возвращено предикатом.  Если предикат всегда возвращает истинное значение,
то функция также вернёт истинное значение.  \code{SOME} возвращает первое не \code{NIL}
значение, возвращённое предикатом, или возвращает ложное значение, если предикат никогда
не вернул истинного значения. \code{NOTANY} возвращает ложное значение, если предикат
возвращает истинное значение, или истинное, если этого не произошло.  А \code{NOTEVERY}
возвращает истинное значение сразу, как только предикат возвращает ложное значение, или
ложное, если предикат всегда возвращал истинное.  Вот примеры проверок для одной
последовательности:

\begin{verbatim}
  (every #'evenp #(1 2 3 4 5))    ==> NIL
  (some #'evenp #(1 2 3 4 5))     ==> T
  (notany #'evenp #(1 2 3 4 5))   ==> NIL
  (notevery #'evenp #(1 2 3 4 5)) ==> T
\end{verbatim}

А эти вызовы выполняют попарное сравнение последовательностей:

\begin{verbatim}
  (every #'> #(1 2 3 4) #(5 4 3 2))    ==> NIL
  (some #'> #(1 2 3 4) #(5 4 3 2))     ==> T
  (notany #'> #(1 2 3 4) #(5 4 3 2))   ==> NIL
  (notevery #'> #(1 2 3 4) #(5 4 3 2)) ==> T
\end{verbatim}

\section{Функции отображения последовательностей}

В заключение рассмотрения функций работы с последовательностями рассмотрим
функции отображения (mapping).  Функция \code{MAP}, подобно функциям-предикатам для
последовательностей, получает функцию нескольких аргументов и несколько
последовательностей.  Но вместо логического значения \code{MAP} возвращает новую
последовательность, содержащую результаты применения функции к элементам
последовательности.  Также как для \code{CONCATENATE} и \code{MERGE}, \code{MAP}
необходимо сообщить тип создаваемой последовательности.

\begin{verbatim}
  (map 'vector #'* #(1 2 3 4 5) #(10 9 8 7 6)) ==> #(10 18 24 28 30)
\end{verbatim}

Функция \code{MAP-INTO} похожа на \code{MAP} за исключением того, что вместо создания
новой последовательности заданного типа, она помещает результаты в последовательность,
заданную в качестве первого аргумента.  Эта последовательность может иметь такой же тип,
как одна из последовательностей, предоставляющих данные для функции.  Например, для
суммирования нескольких векторов (\code{a}, \code{b} и \code{c}) в один, вы должны
написать:

\begin{verbatim}
  (map-into a #'+ a b c)
\end{verbatim}

Если последовательности имеют разную длину, то \code{MAP-INTO} изменяет столько элементов,
сколько присутствует в самой короткой последовательности, включая ту, в которую помещаются
результаты.  Однако, если последовательность будет отображаться в вектор с указателем
заполнения, то число изменяемых элементов будет определяться не указателем заполнения, а
размером вектора.  После вызова \code{MAP-INTO}, указатель заполнения будет установлен
равным количеству изменённых элементов. Однако \code{MAP-INTO} не будет изменять размер
векторов, которые допускают такую операцию.

Напоследок рассмотрим функцию \code{REDUCE},
которая реализует другой вид отображения: она выполняет отображение для
одной последовательности, применяя функцию двух аргументов сначала к первым двум элементам
последовательности, а после первого вызова последовательно применяя её к полученному
результату и следующим элементам.  Таким образом, следующее выражение сложит числа от
единицы до десяти:

\begin{verbatim}
  (reduce #'+ #(1 2 3 4 5 6 7 8 9 10)) ==> 55
\end{verbatim}

\code{REDUCE}~--- удивительно полезная функция: если вам нужно создать из
последовательности одно значение, вполне вероятно, что вы сможете сделать это с помощью
\code{REDUCE}, и такая запись зачастую оказывается весьма лаконичной.
Например, для нахождения максимального значения в последовательности вы можете просто
написать \lstinline!(reduce #'max numbers)!. \code{REDUCE} также принимает полный набор
стандартных именованных аргументов (\code{:key}, \code{:from-end}, \code{:start} и
\code{:end}), а также один, уникальный для \code{REDUCE} (\code{:initial-value}).  Этот
аргумент указывает значение, которое будет логически помещено до первого элемента
последовательности (или после последнего, если вы также зададите \code{:from-end} истинное
значение).

\section{Хэш-таблицы}

Еще одна коллекция общего назначения в Common Lisp~--- хэш-таблица.  В отличие от векторов,
позволяющих осуществлять доступ по целочисленному индексу,
хэш-таблицы позволяют использовать в качестве индексов (ключей) любые объекты.
Добавляя объекты в хэш-таблицу, вы сохраняете их с определённым ключом.  Позднее вы
можете использовать тот же самый ключ для доступа к значению.  Или вы можете связать
с тем же самым ключом новое значение~--- каждый ключ отображается в единственное значение.

Без указания дополнительных аргументов \code{MAKE-HASH-TABLE} создаёт хэш-таблицу, которая
сравнивает ключи с использованием функции \code{EQL}.  Это нормально до тех пор, пока вы
не захотите использовать в качестве ключей строки, поскольку две строки с одинаковым
содержимым не обязательно равны в терминах \code{EQL}.  В таком случае следует
использовать для сравнения функцию \code{EQUAL}: это можно сделать, передав функции
\code{MAKE-HASH-TABLE} символ \code{EQUAL} в качестве именованного аргумента \code{:test}.
Кроме того, для аргумента \code{:test} можно использовать ещё два символа: \code{EQ} и
\code{EQUALP}.  Конечно, эти символы являются именами стандартных функций сравнения
объектов, которые я обсуждал в главе 4.  Однако, в отличие от аргумента \code{:test},
передаваемого функциям работы с последовательностями, аргумент \code{:test} функции
\code{MAKE-HASH-TABLE} не может использовать произвольную функцию~--- допустимы только
значения \code{EQ}, \code{EQL}, \code{EQUAL} и \code{EQUALP}.  Это происходит потому, что
на самом деле хэш-таблице нужно две функции~--- функция сравнения и функция
хэширования, которая получает из ключа числовой хэш-код, способом, совместимым с
функцией сравнения.  Хотя стандарт языка предоставляет
только хэш-таблицы, использующие стандартные функции сравнения, большинство
реализаций позволяют тем или иным образом создавать более тонко настраиваемые
хэш-таблицы.

Функция \code{GETHASH} обеспечивает доступ к элементам хэш-таблиц.  Она принимает два
аргумента: ключ и хэш-таблицу, и возвращает значение, если оно найдено, или \code{NIL} в
противном случае.\footnote{По историческим причинам порядок аргументов \code{GETHASH}
  отличается от порядка аргументов функции \code{ELT}~--- \code{ELT} получает коллекцию в
  качестве первого аргумента, а индекс~--- в качестве второго; а \code{GETHASH} наоборот:
  ключ~--- первым, а коллекцию~--- вторым.}  Например:

\begin{verbatim}
  (defparameter *h* (make-hash-table))
  (gethash 'foo *h*) ==> NIL
  (setf (gethash 'foo *h*) 'quux)
  (gethash 'foo *h*) ==> QUUX
\end{verbatim}

Поскольку \code{GETHASH} возвращает \code{NIL} если ключ не присутствует в таблице, то нет
никакого способа отличить ключ, отсутствующий в таблице, от ключа, по которому хранится
значение \code{NIL}.  Функция \code{GETHASH} решает эту проблему за счёт использования
способа, который мы ещё не обсуждали~--- возврат нескольких значений.  В действительности
\code{GETHASH} возвращает два значения: главное значение~--- значение для указанного ключа
или \code{NIL}.  Дополнительное значение имеет логический тип и указывает, присутствует ли
значение в хэш-таблице.  Из-за способа реализации возврата множественных значений,
дополнительные значения просто отбрасываются, если только пользователь не обрабатывает эту
ситуацию специальным образом, используя средства, которые <<видят>> множественные значения.

Я буду подробно обсуждать возврат множественных значений в главе~\ref{ch:20}, но сейчас я
дам вам лишь общее представление о том, как использовать макрос \code{MULTIPLE-VALUE-BIND}
для получения дополнительных значений, которые возвращает \code{GETHASH}.  Макрос
\code{MULTIPLE-VALUE-BIND} создаёт привязки переменных, как это делает \code{LET},
заполняя их множеством значений, возвращённых вызываемой функцией.

Следующие примеры показывают как вы можете использовать \code{MULTIPLE-VALUE-BIND};
связываемые переменные содержат значение и признак его наличия в таблице:

\begin{lstlisting}  
  (defun show-value (key hash-table)
    (multiple-value-bind (value present) (gethash key hash-table)
      (if present
        (format nil "Значение ~a присутствует в таблице." value)
        (format nil "Значение равно ~a, поскольку ключ не найден." value))))

  (setf (gethash 'bar *h*) nil) ; создаёт ключ со значением NIL

  (show-value 'foo *h*) ==> "Значение QUUX присутствует в таблице."
  (show-value 'bar *h*) ==> "Значение NIL присутствует в таблице."
  (show-value 'baz *h*) ==> "Значение равно NIL , поскольку ключ не найден."
\end{lstlisting}

Поскольку установка значения в \code{NIL} оставляет ключ в таблице, вам понадобится другая
функция для полного удаления пары ключ/значение.  Функция \code{REMHASH} получает такие же
аргументы, как и \code{GETHASH}, и удаляет указанную запись.  Вы также можете полностью
очистить хэш-таблицу с помощью функции \code{CLRHASH}.


\section{Функции для работы с записями в хэш-таблицах}

Common Lisp предоставляет разные способы для работы с записями в хэш-таблицах.  Простейшим
из них является использование функции \code{MAPHASH}.  Так же как и функция \code{MAP},
функция \code{MAPHASH} принимает в качестве аргументов функцию двух аргументов и
хэш-таблицу, и выполняет указанную функцию для каждой пары ключ/значение.  Например, для
распечатки всех пар ключ/значение вы можете использовать такой вызов \code{MAPHASH}:

\begin{verbatim}
  (maphash #'(lambda (k v) (format t "~a => ~a~%" k v)) *h*)
\end{verbatim}

Последствия добавления или удаления записей в хэш-таблице во время прохода по её записям
стандартом не указываются (и скорее всего окажутся печальными), за исключением двух
случаев: вы можете использовать \code{SETF} вместе с \code{GETHASH} для изменения значения
текущей записи, и вы можете использовать \code{REMHASH} для удаления текущей записи.
Например, для удаления всех записей, чьё значение меньше чем десять, вы можете записать
вот так:

\begin{verbatim}
  (maphash #'(lambda (k v) (when (< v 10) (remhash k *h*))) *h*)
\end{verbatim}

Еще один способ итерации по элементам хэш-таблицы~--- использование макроса
\code{LOOP}, который будет описан в главе 22.\footnote{Использование \code{LOOP} для
  работы с хэш-таблицами обычно основывается на использовании более примитивной формы,
  \code{WITH-HASH-TABLE-ITERATOR}, о которой вам не нужно беспокоиться; она была добавлена
  в язык специально для поддержки реализации таких вещей как \code{LOOP} и практически не
  нужна до тех пор, пока вы не соберётесь реализовать совершенно новый способ итерации по
  элементам хэш-таблиц.}  С использованием \code{LOOP} код, реализующий то же, что и первый
пример с \code{MAPHASH}, будет выглядеть вот так:

\begin{lstlisting}  
  (loop for k being the hash-keys in *h* using (hash-value v)
    do (format t "~a => ~a~%" k v))
\end{lstlisting}

Я мог бы рассказать еще очень многое о коллекциях, не являющихся списками, в Common
Lisp.  Например, я совсем не рассказал про многомерные массивы или про библиотеку функций
для работы с битовыми массивами.  Однако того, что я рассказал в этой главе, должно быть
достаточно для основных применений.  Теперь пора взглянуть на
структуру данных, давшую имя языку Lisp~--- списки.

%%% Local Variables: 
%%% mode: latex
%%% TeX-master: "pcl-ru"
%%% TeX-open-quote: "<<"
%%% TeX-close-quote: ">>"
%%% End: 
